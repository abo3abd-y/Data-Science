\documentclass[11pt]{article}

    \usepackage[breakable]{tcolorbox}
    \usepackage{parskip} % Stop auto-indenting (to mimic markdown behaviour)
    

    % Basic figure setup, for now with no caption control since it's done
    % automatically by Pandoc (which extracts ![](path) syntax from Markdown).
    \usepackage{graphicx}
    % Keep aspect ratio if custom image width or height is specified
    \setkeys{Gin}{keepaspectratio}
    % Maintain compatibility with old templates. Remove in nbconvert 6.0
    \let\Oldincludegraphics\includegraphics
    % Ensure that by default, figures have no caption (until we provide a
    % proper Figure object with a Caption API and a way to capture that
    % in the conversion process - todo).
    \usepackage{caption}
    \DeclareCaptionFormat{nocaption}{}
    \captionsetup{format=nocaption,aboveskip=0pt,belowskip=0pt}

    \usepackage{float}
    \floatplacement{figure}{H} % forces figures to be placed at the correct location
    \usepackage{xcolor} % Allow colors to be defined
    \usepackage{enumerate} % Needed for markdown enumerations to work
    \usepackage{geometry} % Used to adjust the document margins
    \usepackage{amsmath} % Equations
    \usepackage{amssymb} % Equations
    \usepackage{textcomp} % defines textquotesingle
    % Hack from http://tex.stackexchange.com/a/47451/13684:
    \AtBeginDocument{%
        \def\PYZsq{\textquotesingle}% Upright quotes in Pygmentized code
    }
    \usepackage{upquote} % Upright quotes for verbatim code
    \usepackage{eurosym} % defines \euro

    \usepackage{iftex}
    \ifPDFTeX
        \usepackage[T1]{fontenc}
        \IfFileExists{alphabeta.sty}{
              \usepackage{alphabeta}
          }{
              \usepackage[mathletters]{ucs}
              \usepackage[utf8x]{inputenc}
          }
    \else
        \usepackage{fontspec}
        \usepackage{unicode-math}
    \fi

    \usepackage{fancyvrb} % verbatim replacement that allows latex
    \usepackage{grffile} % extends the file name processing of package graphics
                         % to support a larger range
    \makeatletter % fix for old versions of grffile with XeLaTeX
    \@ifpackagelater{grffile}{2019/11/01}
    {
      % Do nothing on new versions
    }
    {
      \def\Gread@@xetex#1{%
        \IfFileExists{"\Gin@base".bb}%
        {\Gread@eps{\Gin@base.bb}}%
        {\Gread@@xetex@aux#1}%
      }
    }
    \makeatother
    \usepackage[Export]{adjustbox} % Used to constrain images to a maximum size
    \adjustboxset{max size={0.9\linewidth}{0.9\paperheight}}

    % The hyperref package gives us a pdf with properly built
    % internal navigation ('pdf bookmarks' for the table of contents,
    % internal cross-reference links, web links for URLs, etc.)
    \usepackage{hyperref}
    % The default LaTeX title has an obnoxious amount of whitespace. By default,
    % titling removes some of it. It also provides customization options.
    \usepackage{titling}
    \usepackage{longtable} % longtable support required by pandoc >1.10
    \usepackage{booktabs}  % table support for pandoc > 1.12.2
    \usepackage{array}     % table support for pandoc >= 2.11.3
    \usepackage{calc}      % table minipage width calculation for pandoc >= 2.11.1
    \usepackage[inline]{enumitem} % IRkernel/repr support (it uses the enumerate* environment)
    \usepackage[normalem]{ulem} % ulem is needed to support strikethroughs (\sout)
                                % normalem makes italics be italics, not underlines
    \usepackage{soul}      % strikethrough (\st) support for pandoc >= 3.0.0
    \usepackage{mathrsfs}
    

    
    % Colors for the hyperref package
    \definecolor{urlcolor}{rgb}{0,.145,.698}
    \definecolor{linkcolor}{rgb}{.71,0.21,0.01}
    \definecolor{citecolor}{rgb}{.12,.54,.11}

    % ANSI colors
    \definecolor{ansi-black}{HTML}{3E424D}
    \definecolor{ansi-black-intense}{HTML}{282C36}
    \definecolor{ansi-red}{HTML}{E75C58}
    \definecolor{ansi-red-intense}{HTML}{B22B31}
    \definecolor{ansi-green}{HTML}{00A250}
    \definecolor{ansi-green-intense}{HTML}{007427}
    \definecolor{ansi-yellow}{HTML}{DDB62B}
    \definecolor{ansi-yellow-intense}{HTML}{B27D12}
    \definecolor{ansi-blue}{HTML}{208FFB}
    \definecolor{ansi-blue-intense}{HTML}{0065CA}
    \definecolor{ansi-magenta}{HTML}{D160C4}
    \definecolor{ansi-magenta-intense}{HTML}{A03196}
    \definecolor{ansi-cyan}{HTML}{60C6C8}
    \definecolor{ansi-cyan-intense}{HTML}{258F8F}
    \definecolor{ansi-white}{HTML}{C5C1B4}
    \definecolor{ansi-white-intense}{HTML}{A1A6B2}
    \definecolor{ansi-default-inverse-fg}{HTML}{FFFFFF}
    \definecolor{ansi-default-inverse-bg}{HTML}{000000}

    % common color for the border for error outputs.
    \definecolor{outerrorbackground}{HTML}{FFDFDF}

    % commands and environments needed by pandoc snippets
    % extracted from the output of `pandoc -s`
    \providecommand{\tightlist}{%
      \setlength{\itemsep}{0pt}\setlength{\parskip}{0pt}}
    \DefineVerbatimEnvironment{Highlighting}{Verbatim}{commandchars=\\\{\}}
    % Add ',fontsize=\small' for more characters per line
    \newenvironment{Shaded}{}{}
    \newcommand{\KeywordTok}[1]{\textcolor[rgb]{0.00,0.44,0.13}{\textbf{{#1}}}}
    \newcommand{\DataTypeTok}[1]{\textcolor[rgb]{0.56,0.13,0.00}{{#1}}}
    \newcommand{\DecValTok}[1]{\textcolor[rgb]{0.25,0.63,0.44}{{#1}}}
    \newcommand{\BaseNTok}[1]{\textcolor[rgb]{0.25,0.63,0.44}{{#1}}}
    \newcommand{\FloatTok}[1]{\textcolor[rgb]{0.25,0.63,0.44}{{#1}}}
    \newcommand{\CharTok}[1]{\textcolor[rgb]{0.25,0.44,0.63}{{#1}}}
    \newcommand{\StringTok}[1]{\textcolor[rgb]{0.25,0.44,0.63}{{#1}}}
    \newcommand{\CommentTok}[1]{\textcolor[rgb]{0.38,0.63,0.69}{\textit{{#1}}}}
    \newcommand{\OtherTok}[1]{\textcolor[rgb]{0.00,0.44,0.13}{{#1}}}
    \newcommand{\AlertTok}[1]{\textcolor[rgb]{1.00,0.00,0.00}{\textbf{{#1}}}}
    \newcommand{\FunctionTok}[1]{\textcolor[rgb]{0.02,0.16,0.49}{{#1}}}
    \newcommand{\RegionMarkerTok}[1]{{#1}}
    \newcommand{\ErrorTok}[1]{\textcolor[rgb]{1.00,0.00,0.00}{\textbf{{#1}}}}
    \newcommand{\NormalTok}[1]{{#1}}

    % Additional commands for more recent versions of Pandoc
    \newcommand{\ConstantTok}[1]{\textcolor[rgb]{0.53,0.00,0.00}{{#1}}}
    \newcommand{\SpecialCharTok}[1]{\textcolor[rgb]{0.25,0.44,0.63}{{#1}}}
    \newcommand{\VerbatimStringTok}[1]{\textcolor[rgb]{0.25,0.44,0.63}{{#1}}}
    \newcommand{\SpecialStringTok}[1]{\textcolor[rgb]{0.73,0.40,0.53}{{#1}}}
    \newcommand{\ImportTok}[1]{{#1}}
    \newcommand{\DocumentationTok}[1]{\textcolor[rgb]{0.73,0.13,0.13}{\textit{{#1}}}}
    \newcommand{\AnnotationTok}[1]{\textcolor[rgb]{0.38,0.63,0.69}{\textbf{\textit{{#1}}}}}
    \newcommand{\CommentVarTok}[1]{\textcolor[rgb]{0.38,0.63,0.69}{\textbf{\textit{{#1}}}}}
    \newcommand{\VariableTok}[1]{\textcolor[rgb]{0.10,0.09,0.49}{{#1}}}
    \newcommand{\ControlFlowTok}[1]{\textcolor[rgb]{0.00,0.44,0.13}{\textbf{{#1}}}}
    \newcommand{\OperatorTok}[1]{\textcolor[rgb]{0.40,0.40,0.40}{{#1}}}
    \newcommand{\BuiltInTok}[1]{{#1}}
    \newcommand{\ExtensionTok}[1]{{#1}}
    \newcommand{\PreprocessorTok}[1]{\textcolor[rgb]{0.74,0.48,0.00}{{#1}}}
    \newcommand{\AttributeTok}[1]{\textcolor[rgb]{0.49,0.56,0.16}{{#1}}}
    \newcommand{\InformationTok}[1]{\textcolor[rgb]{0.38,0.63,0.69}{\textbf{\textit{{#1}}}}}
    \newcommand{\WarningTok}[1]{\textcolor[rgb]{0.38,0.63,0.69}{\textbf{\textit{{#1}}}}}


    % Define a nice break command that doesn't care if a line doesn't already
    % exist.
    \def\br{\hspace*{\fill} \\* }
    % Math Jax compatibility definitions
    \def\gt{>}
    \def\lt{<}
    \let\Oldtex\TeX
    \let\Oldlatex\LaTeX
    \renewcommand{\TeX}{\textrm{\Oldtex}}
    \renewcommand{\LaTeX}{\textrm{\Oldlatex}}
    % Document parameters
    % Document title
    \title{hw02}
    
    
    
    
    
    
    
% Pygments definitions
\makeatletter
\def\PY@reset{\let\PY@it=\relax \let\PY@bf=\relax%
    \let\PY@ul=\relax \let\PY@tc=\relax%
    \let\PY@bc=\relax \let\PY@ff=\relax}
\def\PY@tok#1{\csname PY@tok@#1\endcsname}
\def\PY@toks#1+{\ifx\relax#1\empty\else%
    \PY@tok{#1}\expandafter\PY@toks\fi}
\def\PY@do#1{\PY@bc{\PY@tc{\PY@ul{%
    \PY@it{\PY@bf{\PY@ff{#1}}}}}}}
\def\PY#1#2{\PY@reset\PY@toks#1+\relax+\PY@do{#2}}

\@namedef{PY@tok@w}{\def\PY@tc##1{\textcolor[rgb]{0.73,0.73,0.73}{##1}}}
\@namedef{PY@tok@c}{\let\PY@it=\textit\def\PY@tc##1{\textcolor[rgb]{0.24,0.48,0.48}{##1}}}
\@namedef{PY@tok@cp}{\def\PY@tc##1{\textcolor[rgb]{0.61,0.40,0.00}{##1}}}
\@namedef{PY@tok@k}{\let\PY@bf=\textbf\def\PY@tc##1{\textcolor[rgb]{0.00,0.50,0.00}{##1}}}
\@namedef{PY@tok@kp}{\def\PY@tc##1{\textcolor[rgb]{0.00,0.50,0.00}{##1}}}
\@namedef{PY@tok@kt}{\def\PY@tc##1{\textcolor[rgb]{0.69,0.00,0.25}{##1}}}
\@namedef{PY@tok@o}{\def\PY@tc##1{\textcolor[rgb]{0.40,0.40,0.40}{##1}}}
\@namedef{PY@tok@ow}{\let\PY@bf=\textbf\def\PY@tc##1{\textcolor[rgb]{0.67,0.13,1.00}{##1}}}
\@namedef{PY@tok@nb}{\def\PY@tc##1{\textcolor[rgb]{0.00,0.50,0.00}{##1}}}
\@namedef{PY@tok@nf}{\def\PY@tc##1{\textcolor[rgb]{0.00,0.00,1.00}{##1}}}
\@namedef{PY@tok@nc}{\let\PY@bf=\textbf\def\PY@tc##1{\textcolor[rgb]{0.00,0.00,1.00}{##1}}}
\@namedef{PY@tok@nn}{\let\PY@bf=\textbf\def\PY@tc##1{\textcolor[rgb]{0.00,0.00,1.00}{##1}}}
\@namedef{PY@tok@ne}{\let\PY@bf=\textbf\def\PY@tc##1{\textcolor[rgb]{0.80,0.25,0.22}{##1}}}
\@namedef{PY@tok@nv}{\def\PY@tc##1{\textcolor[rgb]{0.10,0.09,0.49}{##1}}}
\@namedef{PY@tok@no}{\def\PY@tc##1{\textcolor[rgb]{0.53,0.00,0.00}{##1}}}
\@namedef{PY@tok@nl}{\def\PY@tc##1{\textcolor[rgb]{0.46,0.46,0.00}{##1}}}
\@namedef{PY@tok@ni}{\let\PY@bf=\textbf\def\PY@tc##1{\textcolor[rgb]{0.44,0.44,0.44}{##1}}}
\@namedef{PY@tok@na}{\def\PY@tc##1{\textcolor[rgb]{0.41,0.47,0.13}{##1}}}
\@namedef{PY@tok@nt}{\let\PY@bf=\textbf\def\PY@tc##1{\textcolor[rgb]{0.00,0.50,0.00}{##1}}}
\@namedef{PY@tok@nd}{\def\PY@tc##1{\textcolor[rgb]{0.67,0.13,1.00}{##1}}}
\@namedef{PY@tok@s}{\def\PY@tc##1{\textcolor[rgb]{0.73,0.13,0.13}{##1}}}
\@namedef{PY@tok@sd}{\let\PY@it=\textit\def\PY@tc##1{\textcolor[rgb]{0.73,0.13,0.13}{##1}}}
\@namedef{PY@tok@si}{\let\PY@bf=\textbf\def\PY@tc##1{\textcolor[rgb]{0.64,0.35,0.47}{##1}}}
\@namedef{PY@tok@se}{\let\PY@bf=\textbf\def\PY@tc##1{\textcolor[rgb]{0.67,0.36,0.12}{##1}}}
\@namedef{PY@tok@sr}{\def\PY@tc##1{\textcolor[rgb]{0.64,0.35,0.47}{##1}}}
\@namedef{PY@tok@ss}{\def\PY@tc##1{\textcolor[rgb]{0.10,0.09,0.49}{##1}}}
\@namedef{PY@tok@sx}{\def\PY@tc##1{\textcolor[rgb]{0.00,0.50,0.00}{##1}}}
\@namedef{PY@tok@m}{\def\PY@tc##1{\textcolor[rgb]{0.40,0.40,0.40}{##1}}}
\@namedef{PY@tok@gh}{\let\PY@bf=\textbf\def\PY@tc##1{\textcolor[rgb]{0.00,0.00,0.50}{##1}}}
\@namedef{PY@tok@gu}{\let\PY@bf=\textbf\def\PY@tc##1{\textcolor[rgb]{0.50,0.00,0.50}{##1}}}
\@namedef{PY@tok@gd}{\def\PY@tc##1{\textcolor[rgb]{0.63,0.00,0.00}{##1}}}
\@namedef{PY@tok@gi}{\def\PY@tc##1{\textcolor[rgb]{0.00,0.52,0.00}{##1}}}
\@namedef{PY@tok@gr}{\def\PY@tc##1{\textcolor[rgb]{0.89,0.00,0.00}{##1}}}
\@namedef{PY@tok@ge}{\let\PY@it=\textit}
\@namedef{PY@tok@gs}{\let\PY@bf=\textbf}
\@namedef{PY@tok@ges}{\let\PY@bf=\textbf\let\PY@it=\textit}
\@namedef{PY@tok@gp}{\let\PY@bf=\textbf\def\PY@tc##1{\textcolor[rgb]{0.00,0.00,0.50}{##1}}}
\@namedef{PY@tok@go}{\def\PY@tc##1{\textcolor[rgb]{0.44,0.44,0.44}{##1}}}
\@namedef{PY@tok@gt}{\def\PY@tc##1{\textcolor[rgb]{0.00,0.27,0.87}{##1}}}
\@namedef{PY@tok@err}{\def\PY@bc##1{{\setlength{\fboxsep}{\string -\fboxrule}\fcolorbox[rgb]{1.00,0.00,0.00}{1,1,1}{\strut ##1}}}}
\@namedef{PY@tok@kc}{\let\PY@bf=\textbf\def\PY@tc##1{\textcolor[rgb]{0.00,0.50,0.00}{##1}}}
\@namedef{PY@tok@kd}{\let\PY@bf=\textbf\def\PY@tc##1{\textcolor[rgb]{0.00,0.50,0.00}{##1}}}
\@namedef{PY@tok@kn}{\let\PY@bf=\textbf\def\PY@tc##1{\textcolor[rgb]{0.00,0.50,0.00}{##1}}}
\@namedef{PY@tok@kr}{\let\PY@bf=\textbf\def\PY@tc##1{\textcolor[rgb]{0.00,0.50,0.00}{##1}}}
\@namedef{PY@tok@bp}{\def\PY@tc##1{\textcolor[rgb]{0.00,0.50,0.00}{##1}}}
\@namedef{PY@tok@fm}{\def\PY@tc##1{\textcolor[rgb]{0.00,0.00,1.00}{##1}}}
\@namedef{PY@tok@vc}{\def\PY@tc##1{\textcolor[rgb]{0.10,0.09,0.49}{##1}}}
\@namedef{PY@tok@vg}{\def\PY@tc##1{\textcolor[rgb]{0.10,0.09,0.49}{##1}}}
\@namedef{PY@tok@vi}{\def\PY@tc##1{\textcolor[rgb]{0.10,0.09,0.49}{##1}}}
\@namedef{PY@tok@vm}{\def\PY@tc##1{\textcolor[rgb]{0.10,0.09,0.49}{##1}}}
\@namedef{PY@tok@sa}{\def\PY@tc##1{\textcolor[rgb]{0.73,0.13,0.13}{##1}}}
\@namedef{PY@tok@sb}{\def\PY@tc##1{\textcolor[rgb]{0.73,0.13,0.13}{##1}}}
\@namedef{PY@tok@sc}{\def\PY@tc##1{\textcolor[rgb]{0.73,0.13,0.13}{##1}}}
\@namedef{PY@tok@dl}{\def\PY@tc##1{\textcolor[rgb]{0.73,0.13,0.13}{##1}}}
\@namedef{PY@tok@s2}{\def\PY@tc##1{\textcolor[rgb]{0.73,0.13,0.13}{##1}}}
\@namedef{PY@tok@sh}{\def\PY@tc##1{\textcolor[rgb]{0.73,0.13,0.13}{##1}}}
\@namedef{PY@tok@s1}{\def\PY@tc##1{\textcolor[rgb]{0.73,0.13,0.13}{##1}}}
\@namedef{PY@tok@mb}{\def\PY@tc##1{\textcolor[rgb]{0.40,0.40,0.40}{##1}}}
\@namedef{PY@tok@mf}{\def\PY@tc##1{\textcolor[rgb]{0.40,0.40,0.40}{##1}}}
\@namedef{PY@tok@mh}{\def\PY@tc##1{\textcolor[rgb]{0.40,0.40,0.40}{##1}}}
\@namedef{PY@tok@mi}{\def\PY@tc##1{\textcolor[rgb]{0.40,0.40,0.40}{##1}}}
\@namedef{PY@tok@il}{\def\PY@tc##1{\textcolor[rgb]{0.40,0.40,0.40}{##1}}}
\@namedef{PY@tok@mo}{\def\PY@tc##1{\textcolor[rgb]{0.40,0.40,0.40}{##1}}}
\@namedef{PY@tok@ch}{\let\PY@it=\textit\def\PY@tc##1{\textcolor[rgb]{0.24,0.48,0.48}{##1}}}
\@namedef{PY@tok@cm}{\let\PY@it=\textit\def\PY@tc##1{\textcolor[rgb]{0.24,0.48,0.48}{##1}}}
\@namedef{PY@tok@cpf}{\let\PY@it=\textit\def\PY@tc##1{\textcolor[rgb]{0.24,0.48,0.48}{##1}}}
\@namedef{PY@tok@c1}{\let\PY@it=\textit\def\PY@tc##1{\textcolor[rgb]{0.24,0.48,0.48}{##1}}}
\@namedef{PY@tok@cs}{\let\PY@it=\textit\def\PY@tc##1{\textcolor[rgb]{0.24,0.48,0.48}{##1}}}

\def\PYZbs{\char`\\}
\def\PYZus{\char`\_}
\def\PYZob{\char`\{}
\def\PYZcb{\char`\}}
\def\PYZca{\char`\^}
\def\PYZam{\char`\&}
\def\PYZlt{\char`\<}
\def\PYZgt{\char`\>}
\def\PYZsh{\char`\#}
\def\PYZpc{\char`\%}
\def\PYZdl{\char`\$}
\def\PYZhy{\char`\-}
\def\PYZsq{\char`\'}
\def\PYZdq{\char`\"}
\def\PYZti{\char`\~}
% for compatibility with earlier versions
\def\PYZat{@}
\def\PYZlb{[}
\def\PYZrb{]}
\makeatother


    % For linebreaks inside Verbatim environment from package fancyvrb.
    \makeatletter
        \newbox\Wrappedcontinuationbox
        \newbox\Wrappedvisiblespacebox
        \newcommand*\Wrappedvisiblespace {\textcolor{red}{\textvisiblespace}}
        \newcommand*\Wrappedcontinuationsymbol {\textcolor{red}{\llap{\tiny$\m@th\hookrightarrow$}}}
        \newcommand*\Wrappedcontinuationindent {3ex }
        \newcommand*\Wrappedafterbreak {\kern\Wrappedcontinuationindent\copy\Wrappedcontinuationbox}
        % Take advantage of the already applied Pygments mark-up to insert
        % potential linebreaks for TeX processing.
        %        {, <, #, %, $, ' and ": go to next line.
        %        _, }, ^, &, >, - and ~: stay at end of broken line.
        % Use of \textquotesingle for straight quote.
        \newcommand*\Wrappedbreaksatspecials {%
            \def\PYGZus{\discretionary{\char`\_}{\Wrappedafterbreak}{\char`\_}}%
            \def\PYGZob{\discretionary{}{\Wrappedafterbreak\char`\{}{\char`\{}}%
            \def\PYGZcb{\discretionary{\char`\}}{\Wrappedafterbreak}{\char`\}}}%
            \def\PYGZca{\discretionary{\char`\^}{\Wrappedafterbreak}{\char`\^}}%
            \def\PYGZam{\discretionary{\char`\&}{\Wrappedafterbreak}{\char`\&}}%
            \def\PYGZlt{\discretionary{}{\Wrappedafterbreak\char`\<}{\char`\<}}%
            \def\PYGZgt{\discretionary{\char`\>}{\Wrappedafterbreak}{\char`\>}}%
            \def\PYGZsh{\discretionary{}{\Wrappedafterbreak\char`\#}{\char`\#}}%
            \def\PYGZpc{\discretionary{}{\Wrappedafterbreak\char`\%}{\char`\%}}%
            \def\PYGZdl{\discretionary{}{\Wrappedafterbreak\char`\$}{\char`\$}}%
            \def\PYGZhy{\discretionary{\char`\-}{\Wrappedafterbreak}{\char`\-}}%
            \def\PYGZsq{\discretionary{}{\Wrappedafterbreak\textquotesingle}{\textquotesingle}}%
            \def\PYGZdq{\discretionary{}{\Wrappedafterbreak\char`\"}{\char`\"}}%
            \def\PYGZti{\discretionary{\char`\~}{\Wrappedafterbreak}{\char`\~}}%
        }
        % Some characters . , ; ? ! / are not pygmentized.
        % This macro makes them "active" and they will insert potential linebreaks
        \newcommand*\Wrappedbreaksatpunct {%
            \lccode`\~`\.\lowercase{\def~}{\discretionary{\hbox{\char`\.}}{\Wrappedafterbreak}{\hbox{\char`\.}}}%
            \lccode`\~`\,\lowercase{\def~}{\discretionary{\hbox{\char`\,}}{\Wrappedafterbreak}{\hbox{\char`\,}}}%
            \lccode`\~`\;\lowercase{\def~}{\discretionary{\hbox{\char`\;}}{\Wrappedafterbreak}{\hbox{\char`\;}}}%
            \lccode`\~`\:\lowercase{\def~}{\discretionary{\hbox{\char`\:}}{\Wrappedafterbreak}{\hbox{\char`\:}}}%
            \lccode`\~`\?\lowercase{\def~}{\discretionary{\hbox{\char`\?}}{\Wrappedafterbreak}{\hbox{\char`\?}}}%
            \lccode`\~`\!\lowercase{\def~}{\discretionary{\hbox{\char`\!}}{\Wrappedafterbreak}{\hbox{\char`\!}}}%
            \lccode`\~`\/\lowercase{\def~}{\discretionary{\hbox{\char`\/}}{\Wrappedafterbreak}{\hbox{\char`\/}}}%
            \catcode`\.\active
            \catcode`\,\active
            \catcode`\;\active
            \catcode`\:\active
            \catcode`\?\active
            \catcode`\!\active
            \catcode`\/\active
            \lccode`\~`\~
        }
    \makeatother

    \let\OriginalVerbatim=\Verbatim
    \makeatletter
    \renewcommand{\Verbatim}[1][1]{%
        %\parskip\z@skip
        \sbox\Wrappedcontinuationbox {\Wrappedcontinuationsymbol}%
        \sbox\Wrappedvisiblespacebox {\FV@SetupFont\Wrappedvisiblespace}%
        \def\FancyVerbFormatLine ##1{\hsize\linewidth
            \vtop{\raggedright\hyphenpenalty\z@\exhyphenpenalty\z@
                \doublehyphendemerits\z@\finalhyphendemerits\z@
                \strut ##1\strut}%
        }%
        % If the linebreak is at a space, the latter will be displayed as visible
        % space at end of first line, and a continuation symbol starts next line.
        % Stretch/shrink are however usually zero for typewriter font.
        \def\FV@Space {%
            \nobreak\hskip\z@ plus\fontdimen3\font minus\fontdimen4\font
            \discretionary{\copy\Wrappedvisiblespacebox}{\Wrappedafterbreak}
            {\kern\fontdimen2\font}%
        }%

        % Allow breaks at special characters using \PYG... macros.
        \Wrappedbreaksatspecials
        % Breaks at punctuation characters . , ; ? ! and / need catcode=\active
        \OriginalVerbatim[#1,codes*=\Wrappedbreaksatpunct]%
    }
    \makeatother

    % Exact colors from NB
    \definecolor{incolor}{HTML}{303F9F}
    \definecolor{outcolor}{HTML}{D84315}
    \definecolor{cellborder}{HTML}{CFCFCF}
    \definecolor{cellbackground}{HTML}{F7F7F7}

    % prompt
    \makeatletter
    \newcommand{\boxspacing}{\kern\kvtcb@left@rule\kern\kvtcb@boxsep}
    \makeatother
    \newcommand{\prompt}[4]{
        {\ttfamily\llap{{\color{#2}[#3]:\hspace{3pt}#4}}\vspace{-\baselineskip}}
    }
    

    
    % Prevent overflowing lines due to hard-to-break entities
    \sloppy
    % Setup hyperref package
    \hypersetup{
      breaklinks=true,  % so long urls are correctly broken across lines
      colorlinks=true,
      urlcolor=urlcolor,
      linkcolor=linkcolor,
      citecolor=citecolor,
      }
    % Slightly bigger margins than the latex defaults
    
    \geometry{verbose,tmargin=1in,bmargin=1in,lmargin=1in,rmargin=1in}
    
    

\begin{document}
    
    \maketitle
    
    

    
    \begin{tcolorbox}[breakable, size=fbox, boxrule=1pt, pad at break*=1mm,colback=cellbackground, colframe=cellborder]
\prompt{In}{incolor}{1}{\boxspacing}
\begin{Verbatim}[commandchars=\\\{\}]
\PY{c+c1}{\PYZsh{} Initialize Otter}
\PY{k+kn}{import} \PY{n+nn}{otter}
\PY{n}{grader} \PY{o}{=} \PY{n}{otter}\PY{o}{.}\PY{n}{Notebook}\PY{p}{(}\PY{l+s+s2}{\PYZdq{}}\PY{l+s+s2}{hw02.ipynb}\PY{l+s+s2}{\PYZdq{}}\PY{p}{)}
\end{Verbatim}
\end{tcolorbox}

    \subsubsection{Importing Libraries and Magic
Commands}\label{importing-libraries-and-magic-commands}

In CSCI 3022, we will be using common Python libraries to help us
process data. By convention, we import all libraries at the very top of
the notebook. There are also a set of standard aliases that are used to
shorten the library names. Below are some of the libraries that you may
encounter throughout this assignment, along with their respective
aliases.

    \begin{tcolorbox}[breakable, size=fbox, boxrule=1pt, pad at break*=1mm,colback=cellbackground, colframe=cellborder]
\prompt{In}{incolor}{2}{\boxspacing}
\begin{Verbatim}[commandchars=\\\{\}]
\PY{k+kn}{import} \PY{n+nn}{numpy} \PY{k}{as} \PY{n+nn}{np}
\PY{k+kn}{import} \PY{n+nn}{pandas} \PY{k}{as} \PY{n+nn}{pd}
\PY{k+kn}{import} \PY{n+nn}{matplotlib}\PY{n+nn}{.}\PY{n+nn}{pyplot} \PY{k}{as} \PY{n+nn}{plt}
\PY{k+kn}{import} \PY{n+nn}{plotly}\PY{n+nn}{.}\PY{n+nn}{express} \PY{k}{as} \PY{n+nn}{px}
\PY{k+kn}{from} \PY{n+nn}{csci3022\PYZus{}utils} \PY{k+kn}{import} \PY{o}{*}
\end{Verbatim}
\end{tcolorbox}

    \section{Homework 2: Prerequisite Review Part 2 \& Exploratory Data
Analysis}\label{homework-2-prerequisite-review-part-2-exploratory-data-analysis}

\subsection{\texorpdfstring{Due Date: \st{Thursday, Jan 25th,} Friday,
Jan 26th 11:59 PM on
Gradescope}{Due Date: Thursday, Jan 25th, Friday, Jan 26th 11:59 PM on Gradescope}}\label{due-date-thursday-jan-25th-friday-jan-26th-1159-pm-on-gradescope}

\subsubsection{Detailed Submission Instructions Are Provided at the end
of this
Notebook}\label{detailed-submission-instructions-are-provided-at-the-end-of-this-notebook}

\subsection{Collaboration Policy}\label{collaboration-policy}

Data science is a collaborative activity. However a key step in learning
and retention is \textbf{creating solutions on your own.}

Below are examples of acceptable vs unacceptable use of resources and
collaboration when doing HW assignments in CSCI 3022.

The following would be some \textbf{examples of cheating} when working
on HW assignments in CSCI 3022. Any of these constitute a
\textbf{violation of the course's collaboration policy and will result
in an F in the course and a trip to the honor council}.

\begin{itemize}
\tightlist
\item
  Consulting web pages that may have a solution to a given homework
  problem or one similar is cheating. However, consulting the class
  notes, and web pages that explain the material taught in class but do
  NOT show a solution to the homework problem in question are
  permissible to view. Clearly, there's a fuzzy line here between a
  valid use of resources and cheating. To avoid this line, one should
  merely consult the course notes, the course textbook, and references
  that contain syntax and/or formulas.
\item
  Copying a segment of code or math solution of three lines or more from
  another student from a printout, handwritten copy, or by looking at
  their computer screen
\item
  Allowing another student to copy a segment of your code or math
  solution of three lines or more
\item
  Taking a copy of another student's work (or a solution found online)
  and then editing that copy
\item
  Reading someone else's solution to a problem on the HW before writing
  your own.
\item
  Asking someone to write all or part of a program or solution for you.
\item
  Asking someone else for the code necessary to fix the error for you,
  other than for simple syntactical errors
\end{itemize}

On the other hand, the following are some \textbf{examples of things
which would NOT usually be considered to be cheating}: - Working on a HW
problem on your own first and then discussing with a classmate a
particular part in the problem solution where you are stuck. After
clarifying any questions you should then continue to write your solution
independently. - Asking someone (or searching online) how a particular
construct in the language works. - Asking someone (or searching online)
how to formulate a particular construct in the language. - Asking
someone for help in finding an error in your program.\\
- Asking someone why a particular construct does not work as you
expected in a given program.

To test whether you are truly doing your own work and retaining what
you've learned you should be able to easily reproduce from scratch and
explain a HW solution that was your own when asked in office hours by a
TA/Instructor or on a quiz/exam.

If you have difficulty in formulating the general solution to a problem
on your own, or you have difficulty in translating that general solution
into a program, it is advisable to see your instructor or teaching
assistant rather than another student as this situation can easily lead
to a, possibly inadvertent, cheating situation.

We are here to help! Visit HW Hours and/or post questions on Piazza!

    \subsection{Grading}\label{grading}

Grading is broken down into autograded answers and manually graded
answers.

For autograded answers, the results of your code are compared to
provided and/or hidden tests.

For manually graded answers you must show and explain all steps. Graders
will evaluate how well you answered the question and/or fulfilled the
requirements of the question.

\subsubsection{Score breakdown}\label{score-breakdown}

\begin{longtable}[]{@{}lll@{}}
\toprule\noalign{}
Question & Points & Grading Type \\
\midrule\noalign{}
\endhead
\bottomrule\noalign{}
\endlastfoot
Question 1a & 3 & autograded \\
Question 1b & 4 & manual \\
Question 1c & 3 & manual \\
Question 2a & 4 & manual \\
Question 2b & 6 & manual \\
Question 3a & 3 & auto \\
Question 3b & 3 & manual \\
Question 3c & 1 & autograded \\
Question 3d & 4 & manual \\
Question 4a & 8 & autograded and manual \\
Question 4b & 4 & autograded \\
Question 4c & 4 & autograded \\
Question 4d & 3 & manual \\
Total & 50 & \\
\end{longtable}

    \subsection{Recommended Readings for this
HW:}\label{recommended-readings-for-this-hw}

\begin{itemize}
\tightlist
\item
  \href{https://canvas.colorado.edu/courses/101142/pages/prerequisite-review-calculus-and-discrete-structures?module_item_id=5144193}{Prerequisite
  Review Resources posted on the Modules in Canvas}
\item
  \href{https://learningds.org/ch/06/pandas_intro.html}{Learning Data
  Science Chapter 6 - All (Subsetting, Aggregating, Joining,
  Transforming)}
\item
  \href{https://learningds.org/ch/08/files_granularity.html\#table-shape-and-granularity}{Learning
  Data Science Section 8.6 (Table Shape and Granularity)}
\end{itemize}

    \subsection{}\label{section}

\subsection{\texorpdfstring{\textbf{Shortcuts:}
\hyperref[p2]{Question 2} \textbar{} \hyperref[p3]{Question 3}
\textbar{} \hyperref[p4]{Question 4}
\textbar{}}{Shortcuts:  \textbar{}  \textbar{}  \textbar{}}}\label{shortcuts-question-2-question-3-question-4}

    

    \subsection{\texorpdfstring{ Question 1: Prerequisites: Discrete
Structures
Review}{ Question 1: Prerequisites: Discrete Structures Review}}\label{question-1-prerequisites-discrete-structures-review}

You will need a solid understanding of the following probability
concepts from Discrete Structures to succeed in this course.

See the
\href{https://canvas.colorado.edu/courses/101142/pages/prerequisite-review-calculus-and-discrete-structures?module_item_id=5144193}{Prerequisite
Review Resources} posted on Canvas if you need to review any of these
key concepts.

    \subsubsection{Question 1a)}\label{question-1a}

A coin is flipped 10 times. How many possible outcomes have exactly 3
heads?

Use code to enter your answer in the cell below.

    \begin{tcolorbox}[breakable, size=fbox, boxrule=1pt, pad at break*=1mm,colback=cellbackground, colframe=cellborder]
\prompt{In}{incolor}{3}{\boxspacing}
\begin{Verbatim}[commandchars=\\\{\}]
\PY{n}{q1a\PYZus{}answer} \PY{o}{=} \PY{l+m+mi}{120} \PY{c+c1}{\PYZsh{}combination with repition not allowed because we are basically saying that order does not really matter here in this question and repitiion is not allowed either}

\PY{c+c1}{\PYZsh{}since it is a coin flip and we are \PYZdq{}picking\PYZdq{} anything and putting it back}
\PY{n}{q1a\PYZus{}answer}
\end{Verbatim}
\end{tcolorbox}

            \begin{tcolorbox}[breakable, size=fbox, boxrule=.5pt, pad at break*=1mm, opacityfill=0]
\prompt{Out}{outcolor}{3}{\boxspacing}
\begin{Verbatim}[commandchars=\\\{\}]
120
\end{Verbatim}
\end{tcolorbox}
        
    \begin{tcolorbox}[breakable, size=fbox, boxrule=1pt, pad at break*=1mm,colback=cellbackground, colframe=cellborder]
\prompt{In}{incolor}{4}{\boxspacing}
\begin{Verbatim}[commandchars=\\\{\}]
\PY{n}{grader}\PY{o}{.}\PY{n}{check}\PY{p}{(}\PY{l+s+s2}{\PYZdq{}}\PY{l+s+s2}{q1a}\PY{l+s+s2}{\PYZdq{}}\PY{p}{)}
\end{Verbatim}
\end{tcolorbox}

            \begin{tcolorbox}[breakable, size=fbox, boxrule=.5pt, pad at break*=1mm, opacityfill=0]
\prompt{Out}{outcolor}{4}{\boxspacing}
\begin{Verbatim}[commandchars=\\\{\}]
q1a results: All test cases passed!
\end{Verbatim}
\end{tcolorbox}
        
    \paragraph{Question 1b)}\label{question-1b}

What is the probability that if I roll two 6-sided dice they add up to
\textbf{at most} \(9\)? Use LaTeX (not code) in the cell directly below
to show all of your steps and fully justify your answer.

    \subsection{Answer}\label{answer}

\subsubsection{\texorpdfstring{\textbf{Part 1: finding how many ways we
can combine two results from the
dice}}{Part 1: finding how many ways we can combine two results from the dice}}\label{part-1-finding-how-many-ways-we-can-combine-two-results-from-the-dice}

First of all, we need to understand how many ways we can combine those
two rolls. Since ordering matters here since, for example, rolling 4 as
the first dice and 2 on the second dice is different than rolling 2 on
the first dice and 4 on the second dice. Thus, we are dealing with
\textbf{permutations}.

Next, we need to know whether repitition is allowed or not. Since we can
pick, for example, 2 on the first dice and 2 on the second dice as
allowed, then \textbf{repitition is allowed}.

Therefore, we can say that we are dealing with permutations with
repitition allowed. Thus, there are \textbf{\(6^2 = 36\)} ways to roll
the two dices.

\subsubsection{\texorpdfstring{\textbf{Part 2: finding how many ways we
can get the dices' sum equal to at most
9}}{Part 2: finding how many ways we can get the dices' sum equal to at most 9}}\label{part-2-finding-how-many-ways-we-can-get-the-dices-sum-equal-to-at-most-9}

Here, we can count those values individually. In the code below, when I
write \((x_1,x_2)\), this just means that \(x_1\) is the outcome of the
first dice while \(x_2\) is the outcome of the second dice. Thus, these
are the ways we can roll the two dices where the sum of their values are
less than or equal to 9:

(1,1), (1,2), (1,3), (1,4), (1,5), (1,6)

(2,1), (2,2), (2,3), (2,4), (2,5), (2,6)

(3,1), (3,2), (3,3), (3,4), (3,5), (3,6)

(4,1), (4,2), (4,3), (4,4), (4,5)

(5,1), (5,2), (5,3), (5,4)

(6,1), (6,2), (6,3)

Thus, from the above list, there are \(30\) ways we can arrange the
dices so that we can add them up to \(9\).

\subsubsection{\texorpdfstring{\textbf{Part 3: Dividing the above two
values
together}}{Part 3: Dividing the above two values together}}\label{part-3-dividing-the-above-two-values-together}

Now, we just divide 30 over 36 to get the probability that if I roll two
6-sided dice they add up to \textbf{at most} \(9\) : \[\frac{30}{36}\]

    1b Answer Check). To check your final answer to 1b, enter the answer you
came up with (just the number) in the cell below. Note that this is just
a built-in public test so you can check your work and determine if you
are on the right track. To receive credit on this problem you must show
all steps in part 1b above using LaTeX and fully justifying your answer
using correct mathematical notation.

    \begin{tcolorbox}[breakable, size=fbox, boxrule=1pt, pad at break*=1mm,colback=cellbackground, colframe=cellborder]
\prompt{In}{incolor}{5}{\boxspacing}
\begin{Verbatim}[commandchars=\\\{\}]
\PY{n}{q1b\PYZus{}answer} \PY{o}{=} \PY{l+m+mi}{30}\PY{o}{/} \PY{l+m+mi}{36} 

\PY{c+c1}{\PYZsh{}6 6 6 5 4 3}

\PY{n+nb}{print}\PY{p}{(}\PY{n}{q1b\PYZus{}answer}\PY{p}{)}
\end{Verbatim}
\end{tcolorbox}

    \begin{Verbatim}[commandchars=\\\{\}]
0.8333333333333334
    \end{Verbatim}

    \begin{tcolorbox}[breakable, size=fbox, boxrule=1pt, pad at break*=1mm,colback=cellbackground, colframe=cellborder]
\prompt{In}{incolor}{6}{\boxspacing}
\begin{Verbatim}[commandchars=\\\{\}]
\PY{n}{grader}\PY{o}{.}\PY{n}{check}\PY{p}{(}\PY{l+s+s2}{\PYZdq{}}\PY{l+s+s2}{q1bcheck}\PY{l+s+s2}{\PYZdq{}}\PY{p}{)}
\end{Verbatim}
\end{tcolorbox}

            \begin{tcolorbox}[breakable, size=fbox, boxrule=.5pt, pad at break*=1mm, opacityfill=0]
\prompt{Out}{outcolor}{6}{\boxspacing}
\begin{Verbatim}[commandchars=\\\{\}]
q1bcheck results: All test cases passed!
\end{Verbatim}
\end{tcolorbox}
        
    \paragraph{Question 1c)}\label{question-1c}

Suppose you uncharacteristically show up to a quiz completely
unprepared. The quiz has 10 questions, each with 5 multiple choice
options. You decide to guess each answer in a completely random way.
What is the probability that you get exactly 3 questions correct? Use
Markdown and LaTeX (not code) in the cell directly below to show all of
your steps and fully justify your answer.

    \subsubsection{\texorpdfstring{\textbf{Part 1: Finding what method we
can use to solve this
problem}}{Part 1: Finding what method we can use to solve this problem}}\label{part-1-finding-what-method-we-can-use-to-solve-this-problem}

Since we are dealing with only two values: either you get the answer
correct or you do not, we will be using the Bernoulli distribution
equation in order to calculate this probability.

\subsubsection{\texorpdfstring{\textbf{Part 2: Using the Bernoulli
distribution
equation}}{Part 2: Using the Bernoulli distribution equation}}\label{part-2-using-the-bernoulli-distribution-equation}

Now, we start our calculations. First, we need to find our \(p\) value
which could be found easily. Since there is four incorrect answers out
of the 5 multiple choice questions, the probability that you will answer
a question incorrectly is \(p = \frac{4}{5}\), which means that the
probability that you will get the answer correctly (if you guess
ofcourse) is: \(1 -p = \frac{1}{5}.\)

Now that we have our \(p\) value, and since we know our \(k\) in this
case (\(k = 7\) since we are getting 7 incorrect questions and 3 correct
answers), we can do the following: \[
\begin{align*}
    P(x)  &= \binom{n}{x} \cdot p^x (1-p)^{n - x} && \text{Initial equation} \\
    &= \binom{10}{7} \cdot (\frac{4}{5}) ^{7} \cdot (\frac{1}{5})^{3} && \text{Plugging in the values with $x = 7$ and $p = \frac{4}{5}$ and $n = 10$ (total number of questions)} \\
    &= \frac{10!}{7! \cdot (3!)} \cdot (\frac{4}{5}) ^{7} \cdot (\frac{1}{5})^{3} && \text{simplify} \\
    &= \frac{8 \cdot 9 \cdot 10}{3!} \cdot (\frac{4}{5}) ^{7} \cdot (\frac{1}{5})^{3} && \text{simplify} \\
    &= 0.2013265900 ... && \text{Final Answer}
\end{align*}
\]

Therefore, the probability that you will get 3 correct answers out of
the 10 multiple-choice questions with 5 options each is
\(0.2013265900 ...\)

    1c Final Answer Check). To check your final answer to 1c, enter the
final answer you came up with in the cell below. Do not round your
answer. Note that this is just a built-in public test so you can check
your work and determine if you are on the right track. To receive credit
on this problem you must show all steps in part 1c above using LaTeX and
fully justifying your answer using correct mathematical notation.

    \begin{tcolorbox}[breakable, size=fbox, boxrule=1pt, pad at break*=1mm,colback=cellbackground, colframe=cellborder]
\prompt{In}{incolor}{7}{\boxspacing}
\begin{Verbatim}[commandchars=\\\{\}]
\PY{n}{q1c\PYZus{}answer} \PY{o}{=} \PY{p}{(}\PY{p}{(}\PY{l+m+mi}{4}\PY{o}{/}\PY{l+m+mi}{5}\PY{p}{)}\PY{o}{*}\PY{o}{*}\PY{p}{(}\PY{l+m+mi}{7}\PY{p}{)}\PY{p}{)} \PY{o}{*} \PY{p}{(}\PY{p}{(}\PY{l+m+mi}{1}\PY{o}{/}\PY{l+m+mi}{5}\PY{p}{)}\PY{o}{*}\PY{o}{*}\PY{p}{(}\PY{l+m+mi}{3}\PY{p}{)}\PY{p}{)} \PY{o}{*} \PY{p}{(}\PY{p}{(}\PY{l+m+mi}{8} \PY{o}{*} \PY{l+m+mi}{9} \PY{o}{*} \PY{l+m+mi}{10}\PY{p}{)}\PY{o}{/}\PY{l+m+mi}{6}\PY{p}{)}  \PY{c+c1}{\PYZsh{}ways of getting 3 correct answers / ways of answering the questions}

\PY{c+c1}{\PYZsh{}i think the 120 is correct but i also  think that the denominator is not correct because i have to do a summation of all values}
\PY{n+nb}{print} \PY{p}{(}\PY{n}{q1c\PYZus{}answer}\PY{p}{)}
\end{Verbatim}
\end{tcolorbox}

    \begin{Verbatim}[commandchars=\\\{\}]
0.2013265920000001
    \end{Verbatim}

    \begin{tcolorbox}[breakable, size=fbox, boxrule=1pt, pad at break*=1mm,colback=cellbackground, colframe=cellborder]
\prompt{In}{incolor}{8}{\boxspacing}
\begin{Verbatim}[commandchars=\\\{\}]
\PY{n}{grader}\PY{o}{.}\PY{n}{check}\PY{p}{(}\PY{l+s+s2}{\PYZdq{}}\PY{l+s+s2}{q1ccheck}\PY{l+s+s2}{\PYZdq{}}\PY{p}{)}
\end{Verbatim}
\end{tcolorbox}

            \begin{tcolorbox}[breakable, size=fbox, boxrule=.5pt, pad at break*=1mm, opacityfill=0]
\prompt{Out}{outcolor}{8}{\boxspacing}
\begin{Verbatim}[commandchars=\\\{\}]
q1ccheck results: All test cases passed!
\end{Verbatim}
\end{tcolorbox}
        
    \hyperref[top]{Back to top}

\subsection{\texorpdfstring{ Question 2: Prerequisites: Calculus
Review}{ Question 2: Prerequisites: Calculus Review}}\label{question-2-prerequisites-calculus-review}

You will need a solid understanding of the following concepts from
Calculus 1 and 2 to succeed in this course. See the
\href{https://canvas.colorado.edu/courses/101142/pages/prerequisite-review-calculus-and-discrete-structures?module_item_id=5144193}{Prerequisite
Review Resources} posted on Canvas if you need to review any of these
key concepts.

\subsubsection{Preliminary: Sums}\label{preliminary-sums}

Here's a recap of some basic algebra written in sigma notation. The
facts are all just applications of the ordinary associative and
distributive properties of addition and multiplication, written
compactly and without the possibly ambiguous ``\ldots{}''. But if you
are ever unsure of whether you're working correctly with a sum, you can
always try writing \(\sum_{i=1}^n a_i\) as \(a_1 + a_2 + \cdots + a_n\)
and see if that helps.

\begin{itemize}
\tightlist
\item
  You can use any reasonable notation for the index over which you are
  summing, just as in Python you can use any reasonable name in
  \texttt{for\ name\ in\ list}. Thus
  \(\sum_{i=1}^n a_i = \sum_{k=1}^n a_k\).
\item
  \(\sum_{i=1}^n (a_i + b_i) = \sum_{i=1}^n a_i + \sum_{i=1}^n b_i\)
\item
  \(\sum_{i=1}^n d = nd\)
\item
  \(\sum_{i=1}^n (ca_i + d) = c\sum_{i=1}^n a_i + nd\)
\end{itemize}

    \subsubsection{Question 2a)}\label{question-2a}

We commonly use sigma notation to compactly write the definition of the
arithmetic mean (commonly known as the average):

\[\bar{x} = \frac{1}{n}\left(x_1+x_2+ ... + x_n \right) = \frac{1}{n}\sum_{i=1}^n x_i\]

The \(i\)th \emph{deviation from average} is the difference
\(x_i - \bar{x}\). Prove that the sum of all these deviations is 0 that
is, prove that \(\sum_{i=1}^n (x_i - \bar{x}) = 0\) (write your full
solution in the box directly below showing all steps and using LaTeX).

    First, we will start with the equation written above and then we will
move on from there:

\[
\begin{align*}
    \sum_{i=1}^n (x_i - \bar{x})  &=  \sum_{i=1}^n (x_i) - \sum_{i=1}^n (\bar{x}) && \text{Expanding the summation} \\
    &= n\bar{x} - n\bar{x} && \text{Since $\bar{x} = \frac{1}{n}\sum_{i=1}^n x_i$ or $n\bar{x} = \sum_{i=1}^n x_i$ and for the second part: $\sum_{i=1}^n d = nd$ for any constant $d$ } \\
    &= 0 && \text{simplify} 
\end{align*}
\]

Thus, we have shown that \(\sum_{i=1}^n (x_i - \bar{x}) = 0\).

    \subsubsection{Question 2b)}\label{question-2b}

Let \(x_1, x_2, \ldots, x_n\) be a list of numbers. You can think of
each index \(i\) as the label of a household, and the entry \(x_i\) as
the annual income of Household \(i\).

Consider the function \[f(c) = \frac{1}{n} \sum_{i=1}^n (x_i-c)^2\]

In this scenario, suppose that our data points \(x_1, x_2, \ldots, x_n\)
are fixed and that \(c\) is the only variable.

Using calculus, determine the value of \(c\) that minimizes \(f(c)\).
You must use calculus to justify that this is indeed a minimum, and not
a maximum.

    \subsection{\texorpdfstring{\textbf{First step: What this
means}}{First step: What this means}}\label{first-step-what-this-means}

First, we need to realize that in order to find value \(c\) that will
minimize \(f(c)\), we need to realize that the derivative of \(f(c)\)
and setting it to zero will give us the maximum/minimum point of the
graph: \[f'(c) = 0\]

\subsection{\texorpdfstring{\textbf{Second step: Solving for \(c\) in
\(f'(c) = 0\):}}{Second step: Solving for c in f\textquotesingle(c) = 0:}}\label{second-step-solving-for-c-in-fc-0}

First, we need to find the derivative of \(f(c)\) and setting it to
zero:

\[
\begin{align*}
    f(c)' &= \frac{dy}{dc}(\frac{1}{n} \sum_{i=1}^n (x_i-c)^2) && \text{Initialization} \\
    &= \frac{dy}{dc}(\frac{1}{n} \sum_{i=1}^n (x_i^2 - 2x_ic + c^2)) && \text{Expanding} \\
    &=  \frac{- 2}{n}\sum_{i=1}^n x_ic + \frac{1}{n}\sum_{i=1}^n 2c && \text{Finding the derivative} \\
    &=  \frac{- 2}{n}\sum_{i=1}^n x_i + \frac{1}{n} \cdot 2cn && \text{Continuing finding the derivative and evaluating the second summation} \\
    &=  -2\bar{x} + \frac{1}{n} \cdot 2cn && \text{Evaluating the first summation. Note: $\bar{x} = \frac{1}{n}\left(x_1+x_2+ ... + x_n \right) = \frac{1}{n}\sum_{i=1}^n x_i$ where $\bar{x}$ is the average} \\
     &=  -2\bar{x} + 2c = 0 && \text{Simplifying and setting the derivative to zero} \\
\end{align*}
\]

Now, let's solve for \(c\):

\[
-2\bar{x} + 2c = 0
\]

\[
2c = 2\bar{x}
\]

\[
c = \bar{x}
\]

This means that the value that minimizes \(c\) is the average of the
household income. We need to verify that this point is indeed the
minimum point and not the maximum.

\[f'(c) = -2\bar{x} + 2c\]

To prove that this is indeed a minimum, we need to verify that the
derivative before \(c = \bar{x}\) is negative and the derivative after
\(c = \bar{x}\) is positive. This is because that a minimum point always
have the derivative right before the point being negative derivative and
the derivative right being after the point is positive. Assuming that
\(\bar{x}\) is positive, let's pick a number that is less than that, say
\(c = -1\):

\[f'(-1) = -2\bar{x} -2\]

Thus, we have shown that the derivative before \(c = \bar{x}\). Let's
show that the derivative after \(c = \bar{x}\) is positive. Let's pick
\(c = \bar{x} + 1\) since it is bigger than \(c = \bar{x}\):

\[f'(\bar{x} + 1) = -2\bar{x} +2(\bar{x} + 1)\]

\[f'(\bar{x} + 1) = -2\bar{x} + 2\bar{x} + 2\]

\[f'(\bar{x} + 1) =  2\]

Thus, we have shown that the derivative after \(c = \bar{x}\) is
positive. Since the derivative before \(c = \bar{x}\) is negative and
after it being positive, then it is verified that \(c = \bar{x}\) is the
value that minimizes \(f(c)\) (it is the minimum point).

    \hyperref[top]{Back to top}

\subsection{\texorpdfstring{ Question 3). Applying Those Prereqs: A
Maximum Likelihood
Estimate}{ Question 3). Applying Those Prereqs: A Maximum Likelihood Estimate}}\label{question-3.-applying-those-prereqs-a-maximum-likelihood-estimate}

    In this problem we're going to apply your calculus and discrete math
prerequisite knowledge to introduce a data science concept called a
maximum likelihood estimate.

Data scientists use coin tossing as a visual image for sampling at
random with replacement from a binary population.

\subsubsection{Question 3a)}\label{question-3a}

A coin that lands heads with chance 0.7 is tossed six times. What is the
chance of the sequence HHHTHT? Assign your answer to the variable
\texttt{p\_HHHTHT}.

    \begin{tcolorbox}[breakable, size=fbox, boxrule=1pt, pad at break*=1mm,colback=cellbackground, colframe=cellborder]
\prompt{In}{incolor}{9}{\boxspacing}
\begin{Verbatim}[commandchars=\\\{\}]
\PY{n}{p\PYZus{}HHHTHT} \PY{o}{=} \PY{p}{(}\PY{p}{(}\PY{l+m+mf}{0.7}\PY{p}{)}\PY{o}{*}\PY{o}{*}\PY{p}{(}\PY{l+m+mi}{4}\PY{p}{)}\PY{p}{)} \PY{o}{*} \PY{p}{(}\PY{p}{(}\PY{l+m+mf}{0.3}\PY{p}{)}\PY{o}{*}\PY{o}{*}\PY{p}{(}\PY{l+m+mi}{2}\PY{p}{)}\PY{p}{)}
\PY{n}{p\PYZus{}HHHTHT}
\end{Verbatim}
\end{tcolorbox}

            \begin{tcolorbox}[breakable, size=fbox, boxrule=.5pt, pad at break*=1mm, opacityfill=0]
\prompt{Out}{outcolor}{9}{\boxspacing}
\begin{Verbatim}[commandchars=\\\{\}]
0.021608999999999996
\end{Verbatim}
\end{tcolorbox}
        
    \begin{tcolorbox}[breakable, size=fbox, boxrule=1pt, pad at break*=1mm,colback=cellbackground, colframe=cellborder]
\prompt{In}{incolor}{10}{\boxspacing}
\begin{Verbatim}[commandchars=\\\{\}]
\PY{n}{grader}\PY{o}{.}\PY{n}{check}\PY{p}{(}\PY{l+s+s2}{\PYZdq{}}\PY{l+s+s2}{3a}\PY{l+s+s2}{\PYZdq{}}\PY{p}{)}
\end{Verbatim}
\end{tcolorbox}

            \begin{tcolorbox}[breakable, size=fbox, boxrule=.5pt, pad at break*=1mm, opacityfill=0]
\prompt{Out}{outcolor}{10}{\boxspacing}
\begin{Verbatim}[commandchars=\\\{\}]
3a results: All test cases passed!
\end{Verbatim}
\end{tcolorbox}
        
    \subsubsection{Question 3b)}\label{question-3b}

I have a coin that lands heads with an unknown probability \(p\).

Suppose I toss it 10 times and get the sequence TTTHTHHTTH.

If you toss this coin 10 times, the chance that you get the sequence
above is a function of \(p\). That function is called the
\emph{likelihood} of the sequence TTTHTHHTTH, so we will call it
\(L(p)\).

    What is \(L(p)\) (i.e.~the likelihood) for the sequence TTTHTHHTTH?

Enter your answer below by setting the \texttt{likelihood} variable
equal to the correct function.

(For example \texttt{likelihood\ =\ sin(p)+2p}, althought that is
definitely an incorrect answer!)

Then run the code below to plot the likelihood function.

    \begin{tcolorbox}[breakable, size=fbox, boxrule=1pt, pad at break*=1mm,colback=cellbackground, colframe=cellborder]
\prompt{In}{incolor}{11}{\boxspacing}
\begin{Verbatim}[commandchars=\\\{\}]
\PY{c+c1}{\PYZsh{}At the top of the notebook we already imported a useful plotting module, matplotlib with alias plt}

\PY{n}{p} \PY{o}{=} \PY{n}{np}\PY{o}{.}\PY{n}{linspace}\PY{p}{(}\PY{l+m+mi}{0}\PY{p}{,} \PY{l+m+mi}{1}\PY{p}{,} \PY{l+m+mi}{100}\PY{p}{)} 
\PY{c+c1}{\PYZsh{}This creates an array of 100 p\PYZhy{}values equally spaced between 0 and 1}

\PY{n}{likelihood} \PY{o}{=} \PY{p}{(}\PY{n}{p}\PY{o}{*}\PY{o}{*}\PY{l+m+mi}{4}\PY{p}{)} \PY{o}{*} \PY{p}{(}\PY{p}{(}\PY{l+m+mi}{1}\PY{o}{\PYZhy{}}\PY{n}{p}\PY{p}{)} \PY{o}{*}\PY{o}{*} \PY{l+m+mi}{6}\PY{p}{)}
\PY{c+c1}{\PYZsh{}Define the likelihood function above}

\PY{n}{plt}\PY{o}{.}\PY{n}{plot}\PY{p}{(}\PY{n}{p}\PY{p}{,} \PY{n}{likelihood}\PY{p}{,} \PY{n}{lw}\PY{o}{=}\PY{l+m+mi}{2}\PY{p}{,} \PY{n}{color}\PY{o}{=}\PY{l+s+s1}{\PYZsq{}}\PY{l+s+s1}{darkblue}\PY{l+s+s1}{\PYZsq{}}\PY{p}{)} 
\PY{c+c1}{\PYZsh{}This plots the likelihood function  }

\PY{n}{plt}\PY{o}{.}\PY{n}{xlabel}\PY{p}{(}\PY{l+s+s1}{\PYZsq{}}\PY{l+s+s1}{\PYZdl{}p\PYZdl{}}\PY{l+s+s1}{\PYZsq{}}\PY{p}{)}
\PY{c+c1}{\PYZsh{}This labels the x axis}

\PY{n}{plt}\PY{o}{.}\PY{n}{ylabel}\PY{p}{(}\PY{l+s+s1}{\PYZsq{}}\PY{l+s+s1}{\PYZdl{}L(p)\PYZdl{}}\PY{l+s+s1}{\PYZsq{}}\PY{p}{)}
\PY{c+c1}{\PYZsh{}This labels the y\PYZhy{}axis}

\PY{n}{plt}\PY{o}{.}\PY{n}{title}\PY{p}{(}\PY{l+s+s1}{\PYZsq{}}\PY{l+s+s1}{Likelihood of TTTHTHHTTH}\PY{l+s+s1}{\PYZsq{}}\PY{p}{)}\PY{p}{;}
\PY{c+c1}{\PYZsh{}This titles the plot}
\end{Verbatim}
\end{tcolorbox}

    \begin{center}
    \adjustimage{max size={0.9\linewidth}{0.9\paperheight}}{hw02_files/hw02_33_0.png}
    \end{center}
    { \hspace*{\fill} \\}
    
    \subsubsection{Question 3c)}\label{question-3c}

The value \(\hat{p}\) at which the likelihood function attains its
maximum is called the \emph{maximum likelihood estimate} (MLE) of \(p\).
Among all values of \(p\), it is the one that makes the observed data
most likely.

Using only your plot above, what is the value of \(\hat{p}\)?

    \begin{tcolorbox}[breakable, size=fbox, boxrule=1pt, pad at break*=1mm,colback=cellbackground, colframe=cellborder]
\prompt{In}{incolor}{12}{\boxspacing}
\begin{Verbatim}[commandchars=\\\{\}]
\PY{n}{p\PYZus{}hat} \PY{o}{=} \PY{l+m+mf}{0.4}
\end{Verbatim}
\end{tcolorbox}

    \begin{tcolorbox}[breakable, size=fbox, boxrule=1pt, pad at break*=1mm,colback=cellbackground, colframe=cellborder]
\prompt{In}{incolor}{13}{\boxspacing}
\begin{Verbatim}[commandchars=\\\{\}]
\PY{n}{grader}\PY{o}{.}\PY{n}{check}\PY{p}{(}\PY{l+s+s2}{\PYZdq{}}\PY{l+s+s2}{3c}\PY{l+s+s2}{\PYZdq{}}\PY{p}{)}
\end{Verbatim}
\end{tcolorbox}

            \begin{tcolorbox}[breakable, size=fbox, boxrule=.5pt, pad at break*=1mm, opacityfill=0]
\prompt{Out}{outcolor}{13}{\boxspacing}
\begin{Verbatim}[commandchars=\\\{\}]
3c results: All test cases passed!
\end{Verbatim}
\end{tcolorbox}
        
    \subsubsection{Question 3d)}\label{question-3d}

Notice the value you found graphically for \(\hat{p}\) above also
intuitively makes sense because it is also the observed proportion of
heads in the given sequence TTTHTHHTTH.

Let's prove what you observed graphically above. That is, let's use
calculus to find \(\hat{p}\).

But \textbf{wait before you start trying to find the value \(p\) where
\(L'(p)=0\) (trust us, the algebra is not pretty\ldots)}

USEFUL TIP: The value \(\hat{p}\) at which the function \(L(p)\) attains
its maximum is the same as the value at which the function \(\ln(L(p))\)
attains its maximum.

This tip is hugely important in data science because many probabilities
are products and the natural log function \texttt{ln} function turns
products into sums. It's \textbf{much simpler to take derivatives of a
sum} than a product.

Thus, to find the value \(p\) where \(L'(p)=0\): - Take the natural log
\texttt{ln} of L(p) - Use properties of logs to rewrite products in
\texttt{ln(L(p))} as sums - Take the derivative of this rewritten
version of \texttt{ln(L(p))} - Solve \texttt{ln(L(p))=0} for p - You
should get the same answer that you found graphically above.

You don't have to check that the value you've found produces a max and
not a min -- we'll spare you that step.

Show all steps in the cell below using Markdown and LaTeX

    \subsection{\texorpdfstring{\textbf{Solution:}}{Solution:}}\label{solution}

From the above information, it is said that
\(L'(p) = \frac{dy}{dp} ln(L(P)) = 0\)

Thus, let's begin by taking the natural log of the function:

\[
\begin{align*}
    ln(L(p))  &=  ln((p^4) \cdot (1-p) ^ 6) && \text{Expanding} \\
    &= ln(p^4) + ln((1-p) ^ 6) && \text{since $\log(ab) = \log(a) + \log(b)$} \\
    &= 4\cdot ln(p) + 6\cdot ln(1-p) && \text{simplify} 
\end{align*}
\]

Now that we took the natural log of the function, we can find the
derivative of this function and set it to zero:

\[
\begin{align*}
    \frac{dy}{dp}(4\cdot ln(p) + 6\cdot ln(1-p))  &=  \frac{4}{p} - \frac{6}{1-p} && \text{Finding the derivative. Note that we used $u = 1 - p$ for the second part when deriving} 
\end{align*}
\]

Set it to zero and solve for \(p\):

\[
\frac{4}{p} - \frac{6}{1-p} = 0
\]

\[
\frac{4}{p} = \frac{6}{1-p}
\]

\[
6p = 4 \cdot (1- p)
\]

\[
6p = 4 - 4p
\]

\[
10p = 4
\]

\[
p = \frac{4}{10} = 0.4
\]

Thus, we have verified that \(\hat{p} = 0.4\).

    \hyperref[top]{Back to top}

\subsection{\texorpdfstring{ QUESTION 4: Evidence for Money
Ball}{ QUESTION 4: Evidence for Money Ball}}\label{question-4-evidence-for-money-ball}

In the book \href{https://en.wikipedia.org/wiki/Moneyball}{MoneyBall},
Michael Lewis documented the introduction of statistics and data science
in selecting players for the Oakland A's. in this problem, we're going
to use two datasets from the \href{http://seanlahman.com/}{Sean Lahman's
Baseball Database} which contains the ``complete batting and pitching
statistics from 1871 to 2022, plus fielding statistics, standings, team
stats, managerial records, post-season data, and more.''

For more details about this data, see documentation.txt in the
\texttt{data} folder.

    \subsection{Exploratory Data Analysis}\label{exploratory-data-analysis}

Throughout this question you will be asked to explore the dataset.\\
You will need familiarity with all the methods and attributes introduced
in Chapter 6 of the assigned reading (and Lessons 2 and 3 of Lecture):
\href{https://learningds.org/ch/06/pandas_intro.html}{Learning Data
Science Chapter 6 - All (Subsetting, Aggregating, Joining,
Transforming)}.

There is frequently more than one way to answer question in Pandas, so
we encourage you to dive in and play around with different approaches.
Unless otherwise indicated, you are welcome to use whatever
methods/attributes/approach that you choose.

    There are two .csv files in the \texttt{data} folder of this HW
assignment: \texttt{teams.csv} and \texttt{salaries.csv}

Import the \texttt{Teams.csv} data into a DataFrame called
\texttt{teams\_df} and take a look at the data.

Import the \texttt{Salaries.csv} data into a DataFrame called
\texttt{salaries\_df} and take a look at the data.

    \begin{tcolorbox}[breakable, size=fbox, boxrule=1pt, pad at break*=1mm,colback=cellbackground, colframe=cellborder]
\prompt{In}{incolor}{14}{\boxspacing}
\begin{Verbatim}[commandchars=\\\{\}]
\PY{n}{teams\PYZus{}df}\PY{o}{=} \PY{n}{pd}\PY{o}{.}\PY{n}{read\PYZus{}csv}\PY{p}{(}\PY{l+s+s2}{\PYZdq{}}\PY{l+s+s2}{data/Teams.csv}\PY{l+s+s2}{\PYZdq{}}\PY{p}{)}
\PY{n}{teams\PYZus{}df}
\end{Verbatim}
\end{tcolorbox}

            \begin{tcolorbox}[breakable, size=fbox, boxrule=.5pt, pad at break*=1mm, opacityfill=0]
\prompt{Out}{outcolor}{14}{\boxspacing}
\begin{Verbatim}[commandchars=\\\{\}]
      yearID lgID teamID franchID divID  Rank    G  Ghome   W    L  {\ldots}   DP  \textbackslash{}
0       1871  NaN    BS1      BNA   NaN     3   31    NaN  20   10  {\ldots}   24
1       1871  NaN    CH1      CNA   NaN     2   28    NaN  19    9  {\ldots}   16
2       1871  NaN    CL1      CFC   NaN     8   29    NaN  10   19  {\ldots}   15
3       1871  NaN    FW1      KEK   NaN     7   19    NaN   7   12  {\ldots}    8
4       1871  NaN    NY2      NNA   NaN     5   33    NaN  16   17  {\ldots}   14
{\ldots}      {\ldots}  {\ldots}    {\ldots}      {\ldots}   {\ldots}   {\ldots}  {\ldots}    {\ldots}  ..  {\ldots}  {\ldots}  {\ldots}
3010    2022   NL    SLN      STL     C     1  162   81.0  93   69  {\ldots}  181
3011    2022   AL    TBA      TBD     E     1  162   81.0  86   76  {\ldots}  110
3012    2022   AL    TEX      TEX     W     4  162   81.0  68   94  {\ldots}  143
3013    2022   AL    TOR      TOR     E     2  162   81.0  92   70  {\ldots}  120
3014    2022   NL    WAS      WSN     E     5  162   81.0  55  107  {\ldots}  126

         FP                     name                          park  \textbackslash{}
0     0.834     Boston Red Stockings           South End Grounds I
1     0.829  Chicago White Stockings       Union Base-Ball Grounds
2     0.818   Cleveland Forest Citys  National Association Grounds
3     0.803     Fort Wayne Kekiongas                Hamilton Field
4     0.840         New York Mutuals      Union Grounds (Brooklyn)
{\ldots}     {\ldots}                      {\ldots}                           {\ldots}
3010  0.989      St. Louis Cardinals             Busch Stadium III
3011  0.985           Tampa Bay Rays               Tropicana Field
3012  0.984            Texas Rangers              Globe Life Field
3013  0.986        Toronto Blue Jays                 Rogers Centre
3014  0.982     Washington Nationals                Nationals Park

      attendance  BPF  PPF  teamIDBR  teamIDlahman45  teamIDretro
0            NaN  103   98       BOS             BS1          BS1
1            NaN  104  102       CHI             CH1          CH1
2            NaN   96  100       CLE             CL1          CL1
3            NaN  101  107       KEK             FW1          FW1
4            NaN   90   88       NYU             NY2          NY2
{\ldots}          {\ldots}  {\ldots}  {\ldots}       {\ldots}             {\ldots}          {\ldots}
3010   3320551.0   94   94       STL             SLN          SLN
3011   1128127.0   95   93       TBR             TBA          TBA
3012   2011361.0  100  101       TEX             TEX          TEX
3013   2653830.0  100  100       TOR             TOR          TOR
3014   2026401.0   94   96       WSN             MON          WAS

[3015 rows x 48 columns]
\end{Verbatim}
\end{tcolorbox}
        
    \begin{tcolorbox}[breakable, size=fbox, boxrule=1pt, pad at break*=1mm,colback=cellbackground, colframe=cellborder]
\prompt{In}{incolor}{15}{\boxspacing}
\begin{Verbatim}[commandchars=\\\{\}]
\PY{n}{salaries\PYZus{}df}\PY{o}{=} \PY{n}{pd}\PY{o}{.}\PY{n}{read\PYZus{}csv}\PY{p}{(}\PY{l+s+s2}{\PYZdq{}}\PY{l+s+s2}{data/Salaries.csv}\PY{l+s+s2}{\PYZdq{}}\PY{p}{)}
\PY{n}{salaries\PYZus{}df}
\end{Verbatim}
\end{tcolorbox}

            \begin{tcolorbox}[breakable, size=fbox, boxrule=.5pt, pad at break*=1mm, opacityfill=0]
\prompt{Out}{outcolor}{15}{\boxspacing}
\begin{Verbatim}[commandchars=\\\{\}]
       yearID teamID lgID   playerID    salary
0        1985    ATL   NL  barkele01    870000
1        1985    ATL   NL  bedrost01    550000
2        1985    ATL   NL  benedbr01    545000
3        1985    ATL   NL   campri01    633333
4        1985    ATL   NL  ceronri01    625000
{\ldots}       {\ldots}    {\ldots}  {\ldots}        {\ldots}       {\ldots}
26423    2016    WAS   NL  strasst01  10400000
26424    2016    WAS   NL  taylomi02    524000
26425    2016    WAS   NL  treinbl01    524900
26426    2016    WAS   NL  werthja01  21733615
26427    2016    WAS   NL  zimmery01  14000000

[26428 rows x 5 columns]
\end{Verbatim}
\end{tcolorbox}
        
    \subsubsection{Question 4a). EDA: Structure, Granularity and
Faithfulness}\label{question-4a.-eda-structure-granularity-and-faithfulness}

Examine the structure, granularity and faithfulness of the datasets.
(Hint: The common utility functions we covered in lecture will be useful
here).

Then answer the following questions:

\begin{itemize}
\item
  i). What does the column \texttt{G} represent in the teams dataset?
  (For a description of the columns, see the documentation in the
  \texttt{data} folder).
\item
  ii). What is the granularity of the \texttt{teams.csv} file?
\item
  iii). What is the granularity of the \texttt{salary.csv} file?
\item
  iv). How many rows and columns are in the teams dataset? Assign your
  answer to the variables \texttt{team\_rows} and \texttt{team\_col}
  below.
\item
  v). How many rows and columns are in the salary dataset? Assign your
  answer to the variables \texttt{salary\_rows} and \texttt{salary\_col}
  below.
\item
  vi). How many entries in the \texttt{teams.csv} file are missing
  Attendance Data? Assign your answer to the variable
  \texttt{missing\_attendance} below.
\end{itemize}

    \subsubsection{Answer Cell for Questions
4a(i)(ii)(iii)}\label{answer-cell-for-questions-4aiiiiii}

In this cell, answer questions 4a(i) - (iii) using Markdown (not code).

\textbf{4a(i) Answer:}

According to the documentation, ``G'' represents the games played by
that team during that year.

\textbf{4a(ii) Answer}:

Basically, each row represents the relevant data every year for each
team in the baseball league since 1871 until 2022, and this data
includes the wins, games played, earned run average, and so much more.

\textbf{4a(iii) Answer}:

In here, the salary data shows the salaries of each player for every
team in the baseball league for every year from 1985 till 2016

In the code cells below justify your answers to part (ii) and (iii) and
then answer parts iv through vi

    \begin{tcolorbox}[breakable, size=fbox, boxrule=1pt, pad at break*=1mm,colback=cellbackground, colframe=cellborder]
\prompt{In}{incolor}{16}{\boxspacing}
\begin{Verbatim}[commandchars=\\\{\}]
\PY{n}{teams\PYZus{}df}\PY{o}{.}\PY{n}{sort\PYZus{}values}\PY{p}{(}\PY{l+s+s2}{\PYZdq{}}\PY{l+s+s2}{Rank}\PY{l+s+s2}{\PYZdq{}}\PY{p}{)}\PY{o}{.}\PY{n}{groupby}\PY{p}{(}\PY{l+s+s2}{\PYZdq{}}\PY{l+s+s2}{yearID}\PY{l+s+s2}{\PYZdq{}}\PY{p}{)}\PY{o}{.}\PY{n}{first}\PY{p}{(}\PY{p}{)}

\PY{c+c1}{\PYZsh{} as you can see, we have shown the number one team in the league since 1871 till 2022 using sort value and groupby}

\PY{c+c1}{\PYZsh{} Show work in this cell justifying your answer to part 4a(ii) (hint: either use .value\PYZus{}counts() or .groupby)}
\end{Verbatim}
\end{tcolorbox}

            \begin{tcolorbox}[breakable, size=fbox, boxrule=.5pt, pad at break*=1mm, opacityfill=0]
\prompt{Out}{outcolor}{16}{\boxspacing}
\begin{Verbatim}[commandchars=\\\{\}]
        lgID teamID franchID divID  Rank    G  Ghome    W   L DivWin  {\ldots}  \textbackslash{}
yearID                                                                {\ldots}
1871    None    PH1      PNA  None     1   28    NaN   21   7   None  {\ldots}
1872    None    BS1      BNA  None     1   48    NaN   39   8   None  {\ldots}
1873    None    BS1      BNA  None     1   60    NaN   43  16   None  {\ldots}
1874    None    BS1      BNA  None     1   71    NaN   52  18   None  {\ldots}
1875    None    BS1      BNA  None     1   82    NaN   71   8   None  {\ldots}
{\ldots}      {\ldots}    {\ldots}      {\ldots}   {\ldots}   {\ldots}  {\ldots}    {\ldots}  {\ldots}  ..    {\ldots}  {\ldots}
2018      NL    MIL      MIL     C     1  163   81.0   96  67      Y  {\ldots}
2019      NL    SLN      STL     C     1  162   81.0   91  71      Y  {\ldots}
2020      AL    TBA      TBD     E     1   60   29.0   40  20      Y  {\ldots}
2021      AL    TBA      TBD     E     1  162   81.0  100  62      Y  {\ldots}
2022      NL    ATL      ATL     E     1  162   81.0  101  61      Y  {\ldots}

         DP     FP                    name                      park  \textbackslash{}
yearID
1871     13  0.845  Philadelphia Athletics  Jefferson Street Grounds
1872     44  0.875    Boston Red Stockings       South End Grounds I
1873     54  0.836    Boston Red Stockings       South End Grounds I
1874     53  0.850    Boston Red Stockings       South End Grounds I
1875     56  0.870    Boston Red Stockings       South End Grounds I
{\ldots}     {\ldots}    {\ldots}                     {\ldots}                       {\ldots}
2018    141  0.982       Milwaukee Brewers               Miller Park
2019    168  0.989     St. Louis Cardinals         Busch Stadium III
2020     52  0.985          Tampa Bay Rays           Tropicana Field
2021    130  0.986          Tampa Bay Rays           Tropicana Field
2022    110  0.987          Atlanta Braves             SunTrust Park

        attendance  BPF  PPF  teamIDBR  teamIDlahman45  teamIDretro
yearID
1871           NaN  102   98       ATH             PH1          PH1
1872           NaN  105  100       BOS             BS1          BS1
1873           NaN  106  100       BOS             BS1          BS1
1874           NaN  105   98       BOS             BS1          BS1
1875           NaN  103   96       BOS             BS1          BS1
{\ldots}            {\ldots}  {\ldots}  {\ldots}       {\ldots}             {\ldots}          {\ldots}
2018     2850875.0  102  101       MIL             ML4          MIL
2019     3480393.0   98   97       STL             SLN          SLN
2020           0.0   96   95       TBR             TBA          TBA
2021      761072.0   92   91       TBR             TBA          TBA
2022     3129931.0  102  100       ATL             ATL          ATL

[152 rows x 47 columns]
\end{Verbatim}
\end{tcolorbox}
        
    \begin{tcolorbox}[breakable, size=fbox, boxrule=1pt, pad at break*=1mm,colback=cellbackground, colframe=cellborder]
\prompt{In}{incolor}{17}{\boxspacing}
\begin{Verbatim}[commandchars=\\\{\}]
\PY{n}{salaries\PYZus{}df}\PY{o}{.}\PY{n}{sort\PYZus{}values}\PY{p}{(}\PY{l+s+s2}{\PYZdq{}}\PY{l+s+s2}{salary}\PY{l+s+s2}{\PYZdq{}}\PY{p}{,} \PY{n}{ascending} \PY{o}{=} \PY{k+kc}{False}\PY{p}{)}\PY{o}{.}\PY{n}{groupby}\PY{p}{(}\PY{l+s+s2}{\PYZdq{}}\PY{l+s+s2}{yearID}\PY{l+s+s2}{\PYZdq{}}\PY{p}{)}\PY{o}{.}\PY{n}{first}\PY{p}{(}\PY{p}{)} 

\PY{c+c1}{\PYZsh{}as you can see, for example, we have shown the highest paid baseball player in the league for each year since 1985 till 2016}

\PY{c+c1}{\PYZsh{} Show work in this cell justifying your answer to part 4a(iii) (hint: either use .value\PYZus{}counts() or .groupby)}
\end{Verbatim}
\end{tcolorbox}

            \begin{tcolorbox}[breakable, size=fbox, boxrule=.5pt, pad at break*=1mm, opacityfill=0]
\prompt{Out}{outcolor}{17}{\boxspacing}
\begin{Verbatim}[commandchars=\\\{\}]
       teamID lgID   playerID    salary
yearID
1985      PHI   NL  schmimi01   2130300
1986      NYN   NL  fostege01   2800000
1987      PHI   NL  schmimi01   2127333
1988      SLN   NL  smithoz01   2340000
1989      LAN   NL  hershor01   2766667
1990      ML4   AL  yountro01   3200000
1991      LAN   NL  strawda01   3800000
1992      NYN   NL  bonilbo01   6100000
1993      NYN   NL  bonilbo01   6200000
1994      NYN   NL  bonilbo01   6300000
1995      DET   AL  fieldce01   9237500
1996      DET   AL  fieldce01   9237500
1997      CHA   AL  belleal01  10000000
1998      FLO   NL  sheffga01  14936667
1999      BAL   AL  belleal01  11949794
2000      LAN   NL  brownke01  15714286
2001      TEX   AL  rodrial01  22000000
2002      TEX   AL  rodrial01  22000000
2003      TEX   AL  rodrial01  22000000
2004      BOS   AL  ramirma02  22500000
2005      NYA   AL  rodrial01  26000000
2006      NYA   AL  rodrial01  21680727
2007      NYA   AL  giambja01  23428571
2008      NYA   AL  rodrial01  28000000
2009      NYA   AL  rodrial01  33000000
2010      NYA   AL  rodrial01  33000000
2011      NYA   AL  rodrial01  32000000
2012      NYA   AL  rodrial01  30000000
2013      NYA   AL  rodrial01  29000000
2014      LAN   NL  greinza01  26000000
2015      LAN   NL  kershcl01  32571000
2016      LAN   NL  kershcl01  33000000
\end{Verbatim}
\end{tcolorbox}
        
    \begin{tcolorbox}[breakable, size=fbox, boxrule=1pt, pad at break*=1mm,colback=cellbackground, colframe=cellborder]
\prompt{In}{incolor}{18}{\boxspacing}
\begin{Verbatim}[commandchars=\\\{\}]
\PY{c+c1}{\PYZsh{} Solution Cell for 4a(iv) and 4a(v)}
\PY{c+c1}{\PYZsh{}Use code to find the number of rows and columns in the teams data.  Do not enter any values by hand.}

\PY{n}{team\PYZus{}rows}\PY{o}{=} \PY{n}{teams\PYZus{}df}\PY{o}{.}\PY{n}{shape}\PY{p}{[}\PY{l+m+mi}{0}\PY{p}{]}

\PY{n}{team\PYZus{}col} \PY{o}{=} \PY{n}{teams\PYZus{}df}\PY{o}{.}\PY{n}{shape}\PY{p}{[}\PY{l+m+mi}{1}\PY{p}{]}

\PY{n}{salary\PYZus{}rows} \PY{o}{=} \PY{n}{salaries\PYZus{}df}\PY{o}{.}\PY{n}{shape}\PY{p}{[}\PY{l+m+mi}{0}\PY{p}{]}

\PY{n}{salary\PYZus{}col} \PY{o}{=} \PY{n}{salaries\PYZus{}df}\PY{o}{.}\PY{n}{shape}\PY{p}{[}\PY{l+m+mi}{1}\PY{p}{]}
\end{Verbatim}
\end{tcolorbox}

    \begin{tcolorbox}[breakable, size=fbox, boxrule=1pt, pad at break*=1mm,colback=cellbackground, colframe=cellborder]
\prompt{In}{incolor}{19}{\boxspacing}
\begin{Verbatim}[commandchars=\\\{\}]
\PY{c+c1}{\PYZsh{} Solution Cell for 4a(vi)}
\PY{c+c1}{\PYZsh{} Use code to find the number of rows in the Teams data that are missing attendance data}

\PY{n}{missing\PYZus{}attendance} \PY{o}{=} \PY{n}{teams\PYZus{}df}\PY{p}{[}\PY{p}{(}\PY{n}{np}\PY{o}{.}\PY{n}{isnan}\PY{p}{(}\PY{n}{teams\PYZus{}df}\PY{p}{[}\PY{l+s+s2}{\PYZdq{}}\PY{l+s+s2}{attendance}\PY{l+s+s2}{\PYZdq{}}\PY{p}{]}\PY{p}{)}\PY{p}{)}\PY{p}{]}\PY{o}{.}\PY{n}{shape}\PY{p}{[}\PY{l+m+mi}{0}\PY{p}{]}
\end{Verbatim}
\end{tcolorbox}

    \begin{tcolorbox}[breakable, size=fbox, boxrule=1pt, pad at break*=1mm,colback=cellbackground, colframe=cellborder]
\prompt{In}{incolor}{20}{\boxspacing}
\begin{Verbatim}[commandchars=\\\{\}]
\PY{n}{grader}\PY{o}{.}\PY{n}{check}\PY{p}{(}\PY{l+s+s2}{\PYZdq{}}\PY{l+s+s2}{q4a}\PY{l+s+s2}{\PYZdq{}}\PY{p}{)}
\end{Verbatim}
\end{tcolorbox}

            \begin{tcolorbox}[breakable, size=fbox, boxrule=.5pt, pad at break*=1mm, opacityfill=0]
\prompt{Out}{outcolor}{20}{\boxspacing}
\begin{Verbatim}[commandchars=\\\{\}]
q4a results: All test cases passed!
\end{Verbatim}
\end{tcolorbox}
        
    \subsubsection{Question 4b) EDA: Examining Scope and Temporality of the
Data}\label{question-4b-eda-examining-scope-and-temporality-of-the-data}

Use code to explore the datasets and answer the questions below:

i). How many unique teams are represented in the \texttt{teams\_df}
DataFrame? Write code to calculate this and assign the answer to the
variable \texttt{unique\_teams}

ii). How many unique teams are represented in the \texttt{salaries\_df}
DataFrame? Write code to calculate this and assign the answer to
\texttt{unique\_teams\_salary}

iii). What years does the \texttt{teams\_df} data span?\\
Write code to calculate this and assign the answers to
\texttt{teams\_startyear} and \texttt{teams\_endyear}

iv). What years does the \texttt{salaries\_df} data span?\\
Write code to calculate this and assign the answers to
\texttt{salaries\_startyear} and \texttt{salaries\_endyear}

    \begin{tcolorbox}[breakable, size=fbox, boxrule=1pt, pad at break*=1mm,colback=cellbackground, colframe=cellborder]
\prompt{In}{incolor}{21}{\boxspacing}
\begin{Verbatim}[commandchars=\\\{\}]
\PY{n}{unique\PYZus{}teams} \PY{o}{=} \PY{n}{teams\PYZus{}df}\PY{p}{[}\PY{l+s+s2}{\PYZdq{}}\PY{l+s+s2}{teamID}\PY{l+s+s2}{\PYZdq{}}\PY{p}{]}\PY{o}{.}\PY{n}{unique}\PY{p}{(}\PY{p}{)}\PY{o}{.}\PY{n}{size}

\PY{c+c1}{\PYZsh{}print out result}
\PY{n}{unique\PYZus{}teams}
\end{Verbatim}
\end{tcolorbox}

            \begin{tcolorbox}[breakable, size=fbox, boxrule=.5pt, pad at break*=1mm, opacityfill=0]
\prompt{Out}{outcolor}{21}{\boxspacing}
\begin{Verbatim}[commandchars=\\\{\}]
149
\end{Verbatim}
\end{tcolorbox}
        
    \begin{tcolorbox}[breakable, size=fbox, boxrule=1pt, pad at break*=1mm,colback=cellbackground, colframe=cellborder]
\prompt{In}{incolor}{22}{\boxspacing}
\begin{Verbatim}[commandchars=\\\{\}]
\PY{n}{unique\PYZus{}teams\PYZus{}salary}\PY{o}{=} \PY{n}{salaries\PYZus{}df}\PY{p}{[}\PY{l+s+s2}{\PYZdq{}}\PY{l+s+s2}{teamID}\PY{l+s+s2}{\PYZdq{}}\PY{p}{]}\PY{o}{.}\PY{n}{unique}\PY{p}{(}\PY{p}{)}\PY{o}{.}\PY{n}{size}

\PY{c+c1}{\PYZsh{}print out result}
\PY{n}{unique\PYZus{}teams\PYZus{}salary}
\end{Verbatim}
\end{tcolorbox}

            \begin{tcolorbox}[breakable, size=fbox, boxrule=.5pt, pad at break*=1mm, opacityfill=0]
\prompt{Out}{outcolor}{22}{\boxspacing}
\begin{Verbatim}[commandchars=\\\{\}]
35
\end{Verbatim}
\end{tcolorbox}
        
    \begin{tcolorbox}[breakable, size=fbox, boxrule=1pt, pad at break*=1mm,colback=cellbackground, colframe=cellborder]
\prompt{In}{incolor}{23}{\boxspacing}
\begin{Verbatim}[commandchars=\\\{\}]
\PY{n}{teams\PYZus{}startyear} \PY{o}{=} \PY{n}{np}\PY{o}{.}\PY{n}{min}\PY{p}{(}\PY{n}{teams\PYZus{}df}\PY{p}{[}\PY{l+s+s2}{\PYZdq{}}\PY{l+s+s2}{yearID}\PY{l+s+s2}{\PYZdq{}}\PY{p}{]}\PY{p}{)}

\PY{n}{teams\PYZus{}endyear} \PY{o}{=} \PY{n}{np}\PY{o}{.}\PY{n}{max}\PY{p}{(}\PY{n}{teams\PYZus{}df}\PY{p}{[}\PY{l+s+s2}{\PYZdq{}}\PY{l+s+s2}{yearID}\PY{l+s+s2}{\PYZdq{}}\PY{p}{]}\PY{p}{)}

\PY{c+c1}{\PYZsh{}print out result}
\PY{p}{[}\PY{n}{teams\PYZus{}startyear}\PY{p}{,} \PY{n}{teams\PYZus{}endyear}\PY{p}{]}
\end{Verbatim}
\end{tcolorbox}

            \begin{tcolorbox}[breakable, size=fbox, boxrule=.5pt, pad at break*=1mm, opacityfill=0]
\prompt{Out}{outcolor}{23}{\boxspacing}
\begin{Verbatim}[commandchars=\\\{\}]
[1871, 2022]
\end{Verbatim}
\end{tcolorbox}
        
    \begin{tcolorbox}[breakable, size=fbox, boxrule=1pt, pad at break*=1mm,colback=cellbackground, colframe=cellborder]
\prompt{In}{incolor}{24}{\boxspacing}
\begin{Verbatim}[commandchars=\\\{\}]
\PY{n}{salaries\PYZus{}startyear} \PY{o}{=} \PY{n}{np}\PY{o}{.}\PY{n}{min}\PY{p}{(}\PY{n}{salaries\PYZus{}df}\PY{p}{[}\PY{l+s+s2}{\PYZdq{}}\PY{l+s+s2}{yearID}\PY{l+s+s2}{\PYZdq{}}\PY{p}{]}\PY{p}{)}

\PY{n}{salaries\PYZus{}endyear} \PY{o}{=} \PY{n}{np}\PY{o}{.}\PY{n}{max}\PY{p}{(}\PY{n}{salaries\PYZus{}df}\PY{p}{[}\PY{l+s+s2}{\PYZdq{}}\PY{l+s+s2}{yearID}\PY{l+s+s2}{\PYZdq{}}\PY{p}{]}\PY{p}{)}

\PY{c+c1}{\PYZsh{}print out result}
\PY{p}{[}\PY{n}{salaries\PYZus{}startyear}\PY{p}{,} \PY{n}{salaries\PYZus{}endyear}\PY{p}{]}
\end{Verbatim}
\end{tcolorbox}

            \begin{tcolorbox}[breakable, size=fbox, boxrule=.5pt, pad at break*=1mm, opacityfill=0]
\prompt{Out}{outcolor}{24}{\boxspacing}
\begin{Verbatim}[commandchars=\\\{\}]
[1985, 2016]
\end{Verbatim}
\end{tcolorbox}
        
    \begin{tcolorbox}[breakable, size=fbox, boxrule=1pt, pad at break*=1mm,colback=cellbackground, colframe=cellborder]
\prompt{In}{incolor}{25}{\boxspacing}
\begin{Verbatim}[commandchars=\\\{\}]
\PY{n}{grader}\PY{o}{.}\PY{n}{check}\PY{p}{(}\PY{l+s+s2}{\PYZdq{}}\PY{l+s+s2}{q4b}\PY{l+s+s2}{\PYZdq{}}\PY{p}{)}
\end{Verbatim}
\end{tcolorbox}

            \begin{tcolorbox}[breakable, size=fbox, boxrule=.5pt, pad at break*=1mm, opacityfill=0]
\prompt{Out}{outcolor}{25}{\boxspacing}
\begin{Verbatim}[commandchars=\\\{\}]
q4b results: All test cases passed!
\end{Verbatim}
\end{tcolorbox}
        
    \subsubsection{Question 4c:}\label{question-4c}

If you haven't already, \textbf{read the assigned textbook readings}
\href{https://learningds.org/ch/06/pandas_subsetting.html\#filtering-rows}{on
Filtering} \href{https://learningds.org/ch/06/pandas_subsetting.html}{on
Extracting Data} and
\href{https://learningds.org/ch/06/pandas_transforming.html}{on
Transforming Data}

\begin{itemize}
\item
  i ). For the data science approach used by the Oakland A's team in
  2002, they used historical baseball data starting in 1998. Filter the
  \texttt{teams\_df} to create a new dataframe
  \texttt{teams\_df\_moneyball} with index \texttt{teamID} and columns
  \texttt{teamID}, \texttt{yearID}, \texttt{name}, \texttt{G},
  \texttt{W}. The new dataframe should only include records where the
  year is greater than or equal to 1998.
\item
  ii). Add a new column called \texttt{winFrac} to the dataframe
  \texttt{teams\_df\_moneyball} that calculates the number of Wins
  divided by the number of Games for a given team in a given year,
  \textbf{rounded to the nearest hundredth}. (Hint: Don't use any for
  loops).
\item
  iii). Use an appropriate extraction operator to determine the Oakland
  A's (``OAK'') fraction of wins in the year 1998 (i.e.~write code below
  that selects this value from the \texttt{teams\_df\_moneyball} table)
  and assign it to the variable \texttt{winFrac\_OAK\_98}.
\end{itemize}

    \begin{tcolorbox}[breakable, size=fbox, boxrule=1pt, pad at break*=1mm,colback=cellbackground, colframe=cellborder]
\prompt{In}{incolor}{26}{\boxspacing}
\begin{Verbatim}[commandchars=\\\{\}]
\PY{c+c1}{\PYZsh{} Create teams\PYZus{}df\PYZus{}moneyball dataframe}


\PY{n}{teams\PYZus{}df\PYZus{}moneyball} \PY{o}{=} \PY{n}{teams\PYZus{}df}\PY{p}{[}\PY{p}{(}\PY{n}{teams\PYZus{}df}\PY{p}{[}\PY{l+s+s2}{\PYZdq{}}\PY{l+s+s2}{yearID}\PY{l+s+s2}{\PYZdq{}}\PY{p}{]} \PY{o}{\PYZgt{}}\PY{o}{=} \PY{l+m+mi}{1998}\PY{p}{)}\PY{p}{]}\PY{o}{.}\PY{n}{loc}\PY{p}{[}\PY{p}{:}\PY{p}{,} \PY{p}{[}\PY{l+s+s2}{\PYZdq{}}\PY{l+s+s2}{teamID}\PY{l+s+s2}{\PYZdq{}}\PY{p}{,} \PY{l+s+s2}{\PYZdq{}}\PY{l+s+s2}{yearID}\PY{l+s+s2}{\PYZdq{}}\PY{p}{,} \PY{l+s+s2}{\PYZdq{}}\PY{l+s+s2}{name}\PY{l+s+s2}{\PYZdq{}}\PY{p}{,} \PY{l+s+s2}{\PYZdq{}}\PY{l+s+s2}{G}\PY{l+s+s2}{\PYZdq{}}\PY{p}{,} \PY{l+s+s2}{\PYZdq{}}\PY{l+s+s2}{W}\PY{l+s+s2}{\PYZdq{}}\PY{p}{]}\PY{p}{]}\PY{o}{.}\PY{n}{set\PYZus{}index}\PY{p}{(}\PY{l+s+s2}{\PYZdq{}}\PY{l+s+s2}{teamID}\PY{l+s+s2}{\PYZdq{}}\PY{p}{)}

\PY{n}{teams\PYZus{}df\PYZus{}moneyball}
\end{Verbatim}
\end{tcolorbox}

            \begin{tcolorbox}[breakable, size=fbox, boxrule=.5pt, pad at break*=1mm, opacityfill=0]
\prompt{Out}{outcolor}{26}{\boxspacing}
\begin{Verbatim}[commandchars=\\\{\}]
        yearID                  name    G    W
teamID
ANA       1998        Anaheim Angels  162   85
ARI       1998  Arizona Diamondbacks  162   65
ATL       1998        Atlanta Braves  162  106
BAL       1998     Baltimore Orioles  162   79
BOS       1998        Boston Red Sox  162   92
{\ldots}        {\ldots}                   {\ldots}  {\ldots}  {\ldots}
SLN       2022   St. Louis Cardinals  162   93
TBA       2022        Tampa Bay Rays  162   86
TEX       2022         Texas Rangers  162   68
TOR       2022     Toronto Blue Jays  162   92
WAS       2022  Washington Nationals  162   55

[750 rows x 4 columns]
\end{Verbatim}
\end{tcolorbox}
        
    \begin{tcolorbox}[breakable, size=fbox, boxrule=1pt, pad at break*=1mm,colback=cellbackground, colframe=cellborder]
\prompt{In}{incolor}{27}{\boxspacing}
\begin{Verbatim}[commandchars=\\\{\}]
\PY{c+c1}{\PYZsh{} Add column called winFrac to the dataframe}


\PY{n}{teams\PYZus{}df\PYZus{}moneyball} \PY{o}{=} \PY{n}{teams\PYZus{}df\PYZus{}moneyball}\PY{o}{.}\PY{n}{assign}\PY{p}{(}\PY{n}{winFrac} \PY{o}{=} \PY{n+nb}{round}\PY{p}{(}\PY{n}{teams\PYZus{}df\PYZus{}moneyball}\PY{p}{[}\PY{l+s+s2}{\PYZdq{}}\PY{l+s+s2}{W}\PY{l+s+s2}{\PYZdq{}}\PY{p}{]}\PY{o}{/}\PY{n}{teams\PYZus{}df\PYZus{}moneyball}\PY{p}{[}\PY{l+s+s2}{\PYZdq{}}\PY{l+s+s2}{G}\PY{l+s+s2}{\PYZdq{}}\PY{p}{]}\PY{p}{,} \PY{l+m+mi}{2}\PY{p}{)}\PY{p}{)}



\PY{c+c1}{\PYZsh{} Print out resulting DataFrame}
\PY{n}{teams\PYZus{}df\PYZus{}moneyball}\PY{o}{.}\PY{n}{head}\PY{p}{(}\PY{p}{)}
\end{Verbatim}
\end{tcolorbox}

            \begin{tcolorbox}[breakable, size=fbox, boxrule=.5pt, pad at break*=1mm, opacityfill=0]
\prompt{Out}{outcolor}{27}{\boxspacing}
\begin{Verbatim}[commandchars=\\\{\}]
        yearID                  name    G    W  winFrac
teamID
ANA       1998        Anaheim Angels  162   85     0.52
ARI       1998  Arizona Diamondbacks  162   65     0.40
ATL       1998        Atlanta Braves  162  106     0.65
BAL       1998     Baltimore Orioles  162   79     0.49
BOS       1998        Boston Red Sox  162   92     0.57
\end{Verbatim}
\end{tcolorbox}
        
    \begin{tcolorbox}[breakable, size=fbox, boxrule=1pt, pad at break*=1mm,colback=cellbackground, colframe=cellborder]
\prompt{In}{incolor}{28}{\boxspacing}
\begin{Verbatim}[commandchars=\\\{\}]
\PY{c+c1}{\PYZsh{} Answer to part 4ci:}

\PY{n}{winFrac\PYZus{}OAK\PYZus{}98}\PY{o}{=} \PY{n}{teams\PYZus{}df\PYZus{}moneyball}\PY{p}{[}\PY{p}{(}\PY{n}{teams\PYZus{}df\PYZus{}moneyball}\PY{p}{[}\PY{l+s+s2}{\PYZdq{}}\PY{l+s+s2}{yearID}\PY{l+s+s2}{\PYZdq{}}\PY{p}{]} \PY{o}{==} \PY{l+m+mi}{1998}\PY{p}{)}\PY{p}{]}\PY{o}{.}\PY{n}{loc}\PY{p}{[}\PY{l+s+s2}{\PYZdq{}}\PY{l+s+s2}{OAK}\PY{l+s+s2}{\PYZdq{}}\PY{p}{,} \PY{p}{:}\PY{p}{]}\PY{p}{[}\PY{l+s+s2}{\PYZdq{}}\PY{l+s+s2}{winFrac}\PY{l+s+s2}{\PYZdq{}}\PY{p}{]}

\PY{n}{winFrac\PYZus{}OAK\PYZus{}98}
\end{Verbatim}
\end{tcolorbox}

            \begin{tcolorbox}[breakable, size=fbox, boxrule=.5pt, pad at break*=1mm, opacityfill=0]
\prompt{Out}{outcolor}{28}{\boxspacing}
\begin{Verbatim}[commandchars=\\\{\}]
0.46
\end{Verbatim}
\end{tcolorbox}
        
    \begin{tcolorbox}[breakable, size=fbox, boxrule=1pt, pad at break*=1mm,colback=cellbackground, colframe=cellborder]
\prompt{In}{incolor}{29}{\boxspacing}
\begin{Verbatim}[commandchars=\\\{\}]
\PY{n}{grader}\PY{o}{.}\PY{n}{check}\PY{p}{(}\PY{l+s+s2}{\PYZdq{}}\PY{l+s+s2}{q4c}\PY{l+s+s2}{\PYZdq{}}\PY{p}{)}
\end{Verbatim}
\end{tcolorbox}

            \begin{tcolorbox}[breakable, size=fbox, boxrule=.5pt, pad at break*=1mm, opacityfill=0]
\prompt{Out}{outcolor}{29}{\boxspacing}
\begin{Verbatim}[commandchars=\\\{\}]
q4c results: All test cases passed!
\end{Verbatim}
\end{tcolorbox}
        
    \subsubsection{EDA: Visualizing Data}\label{eda-visualizing-data}

Let's examine how winFrac has varied over time for the Oakland A's.

    \begin{tcolorbox}[breakable, size=fbox, boxrule=1pt, pad at break*=1mm,colback=cellbackground, colframe=cellborder]
\prompt{In}{incolor}{30}{\boxspacing}
\begin{Verbatim}[commandchars=\\\{\}]
\PY{c+c1}{\PYZsh{}The code below uses the module plotlyexpress, which we imported at the top of the notebook using alias px}
\PY{c+c1}{\PYZsh{}Read the code below to see how to use this function, then run this cell}

\PY{n}{px}\PY{o}{.}\PY{n}{scatter}\PY{p}{(}\PY{n}{teams\PYZus{}df\PYZus{}moneyball}\PY{o}{.}\PY{n}{loc}\PY{p}{[}\PY{l+s+s2}{\PYZdq{}}\PY{l+s+s2}{OAK}\PY{l+s+s2}{\PYZdq{}}\PY{p}{]}\PY{p}{,}\PY{n}{x}\PY{o}{=}\PY{l+s+s2}{\PYZdq{}}\PY{l+s+s2}{yearID}\PY{l+s+s2}{\PYZdq{}}\PY{p}{,} \PY{n}{y}\PY{o}{=}\PY{l+s+s2}{\PYZdq{}}\PY{l+s+s2}{winFrac}\PY{l+s+s2}{\PYZdq{}}\PY{p}{)}
\end{Verbatim}
\end{tcolorbox}

    
    
    
Let's compare Oakland's winFrac to 2 other teams:  Boston Red Sox and the New York Yankees:
    \begin{tcolorbox}[breakable, size=fbox, boxrule=1pt, pad at break*=1mm,colback=cellbackground, colframe=cellborder]
\prompt{In}{incolor}{31}{\boxspacing}
\begin{Verbatim}[commandchars=\\\{\}]
\PY{c+c1}{\PYZsh{}Read the entries below to see how to create this plot:}
\PY{c+c1}{\PYZsh{}Then run this cell}

\PY{n}{px}\PY{o}{.}\PY{n}{scatter}\PY{p}{(}\PY{n}{teams\PYZus{}df\PYZus{}moneyball}\PY{o}{.}\PY{n}{loc}\PY{p}{[}\PY{p}{[}\PY{l+s+s2}{\PYZdq{}}\PY{l+s+s2}{OAK}\PY{l+s+s2}{\PYZdq{}}\PY{p}{,}\PY{l+s+s2}{\PYZdq{}}\PY{l+s+s2}{NYA}\PY{l+s+s2}{\PYZdq{}}\PY{p}{,} \PY{l+s+s2}{\PYZdq{}}\PY{l+s+s2}{BOS}\PY{l+s+s2}{\PYZdq{}}\PY{p}{]}\PY{p}{]}\PY{p}{,}\PY{n}{x}\PY{o}{=}\PY{l+s+s2}{\PYZdq{}}\PY{l+s+s2}{yearID}\PY{l+s+s2}{\PYZdq{}}\PY{p}{,} \PY{n}{y}\PY{o}{=}\PY{l+s+s2}{\PYZdq{}}\PY{l+s+s2}{winFrac}\PY{l+s+s2}{\PYZdq{}}\PY{p}{,}\PY{n}{facet\PYZus{}col} \PY{o}{=} \PY{l+s+s1}{\PYZsq{}}\PY{l+s+s1}{name}\PY{l+s+s1}{\PYZsq{}}\PY{p}{)}
\end{Verbatim}
\end{tcolorbox}

    \begin{Verbatim}[commandchars=\\\{\}]
/Users/abdullah/Library/Python/3.9/lib/python/site-
packages/plotly/express/\_core.py:2065: FutureWarning:

When grouping with a length-1 list-like, you will need to pass a length-1 tuple
to get\_group in a future version of pandas. Pass `(name,)` instead of `name` to
silence this warning.

    \end{Verbatim}

    
    
    \textbf{Observations:} In the cell below write down at least 3
observations you can make from these plots.

    First, I noticed that OAK peaked in terms of win fraction around 2022
but since then it has been decreasing generally.

Also, I noticed that New York Yankess preserved their win fraction
throughout the time period shown which means that its performance has
been stable.

Lastly, we have Boston Red Sox that also preserved its win frac
although, very slowly, its win frac began to decrease in the last few
years.

    \textbf{Payrolls Per Team} An interesting part of this story also takes
into account the salary differences between players on these teams.
We'll examine this further in lecture and in our TA-led discussion!

    \subsection{Congratulations! You have finished Homework
2!}\label{congratulations-you-have-finished-homework-2}

    If you discussed this assignment with any other students in the class
(in a manner that is acceptable as described by the Collaboration policy
above) please \textbf{include their names} here:

\textbf{Collaborators}: \emph{list collaborators here}

    If while completing this assignment you reference any websites other
than those linked in this assignment or provided on Canvas please list
those references here:

\textbf{External references}: \emph{list any websites you referenced}

    \subsubsection{Submission Instructions}\label{submission-instructions}

Before proceeding any further, \textbf{save this notebook.}

After running the \texttt{grader.export()} cell provided below,
\textbf{2 files will be created}: a zip file and pdf file. You can
download them using the links provided below OR by finding them in the
same folder where this juptyer notebook resides in your JuptyerHub.

To receive credit on this assignment, \textbf{you must submit BOTH of
these files to their respective Gradescope portals:}

\begin{itemize}
\item
  \textbf{Homework 2 Autograded}: Submit the zip file that is output by
  the \texttt{grader.export()} cell below to the HW2 Autograded
  assignment in Gradescope.
\item
  \textbf{Homework 2 Manually Graded}: Submit your hw02.PDF to the HW2
  Manually Graded assignment in Gradescope. \textbf{It is your
  responsibility to fully review your PDF file before submitting and
  make sure that all your lines of code are visible and any LaTeX has
  correctly compiled and is fully viewable.} \textbf{YOU MUST SELECT THE
  PAGES CORRESPONDING TO EACH QUESTION WHEN YOU UPLOAD TO GRADESCOPE.}
  If not, you will lose points.
\end{itemize}

\textbf{TROUBLESHOOTING TIPS} If you are having any issues compiling
your assignment, please
\href{https://docs.google.com/document/d/1ndr3Wj1PSF5qzlLMaBJznwh6QGeEXjd5TAJ6nf9EJvo/edit?usp=sharing}{read
through these troubleshooting tips first}, then post any questions on
Piazza.

\textbf{You are responsible for ensuring your submission follows our
requirements. We will not be granting regrade requests nor extensions to
submissions that don't follow instructions.} If you encounter any
difficulties with submission, please don't hesitate to reach out to
staff prior to the deadline.

    \subsection{Submission}\label{submission}

Make sure you have run all cells in your notebook in order before
running the cell below, so that all images/graphs appear in the output.
The cell below will generate a zip file for you to submit.
\textbf{Please save before exporting!}

AFTER running the cell below, click on this link to download the PDF to
upload to Gradescope. There will be a separate link that appears after
running the cell below with a link to download the zip file to upload to
Gradescope.

    \begin{tcolorbox}[breakable, size=fbox, boxrule=1pt, pad at break*=1mm,colback=cellbackground, colframe=cellborder]
\prompt{In}{incolor}{38}{\boxspacing}
\begin{Verbatim}[commandchars=\\\{\}]
\PY{c+c1}{\PYZsh{} Save your notebook first, then run this cell to export your submission.}
\PY{n}{grader}\PY{o}{.}\PY{n}{export}\PY{p}{(}\PY{n}{run\PYZus{}tests}\PY{o}{=}\PY{k+kc}{True}\PY{p}{)}
\end{Verbatim}
\end{tcolorbox}

    \begin{Verbatim}[commandchars=\\\{\}]
/Users/abdullah/Library/Python/3.9/share/jupyter/nbconvert/templates/latex/displ
ay\_priority.j2:32: UserWarning:

Your element with mimetype(s) dict\_keys(['application/vnd.plotly.v1+json']) is
not able to be represented.

/Users/abdullah/Library/Python/3.9/lib/python/site-
packages/otter/check/notebook.py:470: UserWarning:

Could not locate a PDF to include

    \end{Verbatim}

    \begin{Verbatim}[commandchars=\\\{\}, frame=single, framerule=2mm, rulecolor=\color{outerrorbackground}]
\textcolor{ansi-red}{---------------------------------------------------------------------------}
\textcolor{ansi-red}{LatexFailed}                               Traceback (most recent call last)
File \textcolor{ansi-green}{\textasciitilde{}/Library/Python/3.9/lib/python/site-packages/otter/export/exporters/via\_latex.py:66}, in \textcolor{ansi-cyan}{PDFViaLatexExporter.convert\_notebook}\textcolor{ansi-blue}{(cls, nb\_path, dest, xecjk, **kwargs)}
\textcolor{ansi-green-intense}{\textbf{     64}}         output\_file\def\tcRGB{\textcolor[RGB]}\expandafter\tcRGB\expandafter{\detokenize{98,98,98}}{.}write(latex\_output[\def\tcRGB{\textcolor[RGB]}\expandafter\tcRGB\expandafter{\detokenize{98,98,98}}{0}])
\textcolor{ansi-green}{---> 66} pdf\_output \def\tcRGB{\textcolor[RGB]}\expandafter\tcRGB\expandafter{\detokenize{98,98,98}}{=} \setlength{\fboxsep}{0pt}\colorbox{ansi-yellow}{nbconvert\strut}\def\tcRGB{\textcolor[RGB]}\expandafter\tcRGB\expandafter{\detokenize{98,98,98}}{\setlength{\fboxsep}{0pt}\colorbox{ansi-yellow}{.\strut}}\setlength{\fboxsep}{0pt}\colorbox{ansi-yellow}{export\strut}\setlength{\fboxsep}{0pt}\colorbox{ansi-yellow}{(\strut}\setlength{\fboxsep}{0pt}\colorbox{ansi-yellow}{pdf\_exporter\strut}\setlength{\fboxsep}{0pt}\colorbox{ansi-yellow}{,\strut}\setlength{\fboxsep}{0pt}\colorbox{ansi-yellow}{ \strut}\setlength{\fboxsep}{0pt}\colorbox{ansi-yellow}{nb\strut}\setlength{\fboxsep}{0pt}\colorbox{ansi-yellow}{)\strut}
\textcolor{ansi-green-intense}{\textbf{     67}} \def\tcRGB{\textcolor[RGB]}\expandafter\tcRGB\expandafter{\detokenize{0,135,0}}{\textbf{with}} \def\tcRGB{\textcolor[RGB]}\expandafter\tcRGB\expandafter{\detokenize{0,135,0}}{open}(dest, \def\tcRGB{\textcolor[RGB]}\expandafter\tcRGB\expandafter{\detokenize{175,0,0}}{"}\def\tcRGB{\textcolor[RGB]}\expandafter\tcRGB\expandafter{\detokenize{175,0,0}}{wb}\def\tcRGB{\textcolor[RGB]}\expandafter\tcRGB\expandafter{\detokenize{175,0,0}}{"}) \def\tcRGB{\textcolor[RGB]}\expandafter\tcRGB\expandafter{\detokenize{0,135,0}}{\textbf{as}} output\_file:

File \textcolor{ansi-green}{\textasciitilde{}/Library/Python/3.9/lib/python/site-packages/nbconvert/exporters/base.py:86}, in \textcolor{ansi-cyan}{export}\textcolor{ansi-blue}{(exporter, nb, **kw)}
\textcolor{ansi-green-intense}{\textbf{     85}} \def\tcRGB{\textcolor[RGB]}\expandafter\tcRGB\expandafter{\detokenize{0,135,0}}{\textbf{if}} \def\tcRGB{\textcolor[RGB]}\expandafter\tcRGB\expandafter{\detokenize{0,135,0}}{isinstance}(nb, NotebookNode):
\textcolor{ansi-green}{---> 86}     output, resources \def\tcRGB{\textcolor[RGB]}\expandafter\tcRGB\expandafter{\detokenize{98,98,98}}{=} \setlength{\fboxsep}{0pt}\colorbox{ansi-yellow}{exporter\_instance\strut}\def\tcRGB{\textcolor[RGB]}\expandafter\tcRGB\expandafter{\detokenize{98,98,98}}{\setlength{\fboxsep}{0pt}\colorbox{ansi-yellow}{.\strut}}\setlength{\fboxsep}{0pt}\colorbox{ansi-yellow}{from\_notebook\_node\strut}\setlength{\fboxsep}{0pt}\colorbox{ansi-yellow}{(\strut}\setlength{\fboxsep}{0pt}\colorbox{ansi-yellow}{nb\strut}\setlength{\fboxsep}{0pt}\colorbox{ansi-yellow}{,\strut}\setlength{\fboxsep}{0pt}\colorbox{ansi-yellow}{ \strut}\setlength{\fboxsep}{0pt}\colorbox{ansi-yellow}{resources\strut}\setlength{\fboxsep}{0pt}\colorbox{ansi-yellow}{)\strut}
\textcolor{ansi-green-intense}{\textbf{     87}} \def\tcRGB{\textcolor[RGB]}\expandafter\tcRGB\expandafter{\detokenize{0,135,0}}{\textbf{elif}} \def\tcRGB{\textcolor[RGB]}\expandafter\tcRGB\expandafter{\detokenize{0,135,0}}{isinstance}(nb, (\def\tcRGB{\textcolor[RGB]}\expandafter\tcRGB\expandafter{\detokenize{0,135,0}}{str},)):

File \textcolor{ansi-green}{\textasciitilde{}/Library/Python/3.9/lib/python/site-packages/nbconvert/exporters/pdf.py:197}, in \textcolor{ansi-cyan}{PDFExporter.from\_notebook\_node}\textcolor{ansi-blue}{(self, nb, resources, **kw)}
\textcolor{ansi-green-intense}{\textbf{    196}} \def\tcRGB{\textcolor[RGB]}\expandafter\tcRGB\expandafter{\detokenize{0,135,0}}{self}\def\tcRGB{\textcolor[RGB]}\expandafter\tcRGB\expandafter{\detokenize{98,98,98}}{.}log\def\tcRGB{\textcolor[RGB]}\expandafter\tcRGB\expandafter{\detokenize{98,98,98}}{.}info(\def\tcRGB{\textcolor[RGB]}\expandafter\tcRGB\expandafter{\detokenize{175,0,0}}{"}\def\tcRGB{\textcolor[RGB]}\expandafter\tcRGB\expandafter{\detokenize{175,0,0}}{Building PDF}\def\tcRGB{\textcolor[RGB]}\expandafter\tcRGB\expandafter{\detokenize{175,0,0}}{"})
\textcolor{ansi-green}{--> 197} \def\tcRGB{\textcolor[RGB]}\expandafter\tcRGB\expandafter{\detokenize{0,135,0}}{\setlength{\fboxsep}{0pt}\colorbox{ansi-yellow}{self\strut}}\def\tcRGB{\textcolor[RGB]}\expandafter\tcRGB\expandafter{\detokenize{98,98,98}}{\setlength{\fboxsep}{0pt}\colorbox{ansi-yellow}{.\strut}}\setlength{\fboxsep}{0pt}\colorbox{ansi-yellow}{run\_latex\strut}\setlength{\fboxsep}{0pt}\colorbox{ansi-yellow}{(\strut}\setlength{\fboxsep}{0pt}\colorbox{ansi-yellow}{tex\_file\strut}\setlength{\fboxsep}{0pt}\colorbox{ansi-yellow}{)\strut}
\textcolor{ansi-green-intense}{\textbf{    198}} \def\tcRGB{\textcolor[RGB]}\expandafter\tcRGB\expandafter{\detokenize{0,135,0}}{\textbf{if}} \def\tcRGB{\textcolor[RGB]}\expandafter\tcRGB\expandafter{\detokenize{0,135,0}}{self}\def\tcRGB{\textcolor[RGB]}\expandafter\tcRGB\expandafter{\detokenize{98,98,98}}{.}run\_bib(tex\_file):

File \textcolor{ansi-green}{\textasciitilde{}/Library/Python/3.9/lib/python/site-packages/nbconvert/exporters/pdf.py:166}, in \textcolor{ansi-cyan}{PDFExporter.run\_latex}\textcolor{ansi-blue}{(self, filename, raise\_on\_failure)}
\textcolor{ansi-green-intense}{\textbf{    164}}     \def\tcRGB{\textcolor[RGB]}\expandafter\tcRGB\expandafter{\detokenize{0,135,0}}{self}\def\tcRGB{\textcolor[RGB]}\expandafter\tcRGB\expandafter{\detokenize{98,98,98}}{.}log\def\tcRGB{\textcolor[RGB]}\expandafter\tcRGB\expandafter{\detokenize{98,98,98}}{.}critical(\def\tcRGB{\textcolor[RGB]}\expandafter\tcRGB\expandafter{\detokenize{175,0,0}}{"}\def\tcRGB{\textcolor[RGB]}\expandafter\tcRGB\expandafter{\detokenize{175,95,135}}{\textbf{\%s}}\def\tcRGB{\textcolor[RGB]}\expandafter\tcRGB\expandafter{\detokenize{175,0,0}}{ failed: }\def\tcRGB{\textcolor[RGB]}\expandafter\tcRGB\expandafter{\detokenize{175,95,135}}{\textbf{\%s}}\def\tcRGB{\textcolor[RGB]}\expandafter\tcRGB\expandafter{\detokenize{175,95,0}}{\textbf{\textbackslash{}n}}\def\tcRGB{\textcolor[RGB]}\expandafter\tcRGB\expandafter{\detokenize{175,95,135}}{\textbf{\%s}}\def\tcRGB{\textcolor[RGB]}\expandafter\tcRGB\expandafter{\detokenize{175,0,0}}{"}, command[\def\tcRGB{\textcolor[RGB]}\expandafter\tcRGB\expandafter{\detokenize{98,98,98}}{0}], command, out)
\textcolor{ansi-green}{--> 166} \def\tcRGB{\textcolor[RGB]}\expandafter\tcRGB\expandafter{\detokenize{0,135,0}}{\textbf{return}} \def\tcRGB{\textcolor[RGB]}\expandafter\tcRGB\expandafter{\detokenize{0,135,0}}{\setlength{\fboxsep}{0pt}\colorbox{ansi-yellow}{self\strut}}\def\tcRGB{\textcolor[RGB]}\expandafter\tcRGB\expandafter{\detokenize{98,98,98}}{\setlength{\fboxsep}{0pt}\colorbox{ansi-yellow}{.\strut}}\setlength{\fboxsep}{0pt}\colorbox{ansi-yellow}{run\_command\strut}\setlength{\fboxsep}{0pt}\colorbox{ansi-yellow}{(\strut}
\textcolor{ansi-green-intense}{\textbf{    167}} \setlength{\fboxsep}{0pt}\colorbox{ansi-yellow}{    \strut}\def\tcRGB{\textcolor[RGB]}\expandafter\tcRGB\expandafter{\detokenize{0,135,0}}{\setlength{\fboxsep}{0pt}\colorbox{ansi-yellow}{self\strut}}\def\tcRGB{\textcolor[RGB]}\expandafter\tcRGB\expandafter{\detokenize{98,98,98}}{\setlength{\fboxsep}{0pt}\colorbox{ansi-yellow}{.\strut}}\setlength{\fboxsep}{0pt}\colorbox{ansi-yellow}{latex\_command\strut}\setlength{\fboxsep}{0pt}\colorbox{ansi-yellow}{,\strut}\setlength{\fboxsep}{0pt}\colorbox{ansi-yellow}{ \strut}\setlength{\fboxsep}{0pt}\colorbox{ansi-yellow}{filename\strut}\setlength{\fboxsep}{0pt}\colorbox{ansi-yellow}{,\strut}\setlength{\fboxsep}{0pt}\colorbox{ansi-yellow}{ \strut}\def\tcRGB{\textcolor[RGB]}\expandafter\tcRGB\expandafter{\detokenize{0,135,0}}{\setlength{\fboxsep}{0pt}\colorbox{ansi-yellow}{self\strut}}\def\tcRGB{\textcolor[RGB]}\expandafter\tcRGB\expandafter{\detokenize{98,98,98}}{\setlength{\fboxsep}{0pt}\colorbox{ansi-yellow}{.\strut}}\setlength{\fboxsep}{0pt}\colorbox{ansi-yellow}{latex\_count\strut}\setlength{\fboxsep}{0pt}\colorbox{ansi-yellow}{,\strut}\setlength{\fboxsep}{0pt}\colorbox{ansi-yellow}{ \strut}\setlength{\fboxsep}{0pt}\colorbox{ansi-yellow}{log\_error\strut}\setlength{\fboxsep}{0pt}\colorbox{ansi-yellow}{,\strut}\setlength{\fboxsep}{0pt}\colorbox{ansi-yellow}{ \strut}\setlength{\fboxsep}{0pt}\colorbox{ansi-yellow}{raise\_on\_failure\strut}
\textcolor{ansi-green-intense}{\textbf{    168}} \setlength{\fboxsep}{0pt}\colorbox{ansi-yellow}{)\strut}

File \textcolor{ansi-green}{\textasciitilde{}/Library/Python/3.9/lib/python/site-packages/nbconvert/exporters/pdf.py:156}, in \textcolor{ansi-cyan}{PDFExporter.run\_command}\textcolor{ansi-blue}{(self, command\_list, filename, count, log\_function, raise\_on\_failure)}
\textcolor{ansi-green-intense}{\textbf{    155}}     msg \def\tcRGB{\textcolor[RGB]}\expandafter\tcRGB\expandafter{\detokenize{98,98,98}}{=} \def\tcRGB{\textcolor[RGB]}\expandafter\tcRGB\expandafter{\detokenize{175,0,0}}{f}\def\tcRGB{\textcolor[RGB]}\expandafter\tcRGB\expandafter{\detokenize{175,0,0}}{'}\def\tcRGB{\textcolor[RGB]}\expandafter\tcRGB\expandafter{\detokenize{175,0,0}}{Failed to run }\def\tcRGB{\textcolor[RGB]}\expandafter\tcRGB\expandafter{\detokenize{175,0,0}}{"}\def\tcRGB{\textcolor[RGB]}\expandafter\tcRGB\expandafter{\detokenize{175,95,135}}{\textbf{\{}}command\def\tcRGB{\textcolor[RGB]}\expandafter\tcRGB\expandafter{\detokenize{175,95,135}}{\textbf{\}}}\def\tcRGB{\textcolor[RGB]}\expandafter\tcRGB\expandafter{\detokenize{175,0,0}}{"}\def\tcRGB{\textcolor[RGB]}\expandafter\tcRGB\expandafter{\detokenize{175,0,0}}{ command:}\def\tcRGB{\textcolor[RGB]}\expandafter\tcRGB\expandafter{\detokenize{175,95,0}}{\textbf{\textbackslash{}n}}\def\tcRGB{\textcolor[RGB]}\expandafter\tcRGB\expandafter{\detokenize{175,95,135}}{\textbf{\{}}out\_str\def\tcRGB{\textcolor[RGB]}\expandafter\tcRGB\expandafter{\detokenize{175,95,135}}{\textbf{\}}}\def\tcRGB{\textcolor[RGB]}\expandafter\tcRGB\expandafter{\detokenize{175,0,0}}{'}
\textcolor{ansi-green}{--> 156}     \def\tcRGB{\textcolor[RGB]}\expandafter\tcRGB\expandafter{\detokenize{0,135,0}}{\textbf{raise}} raise\_on\_failure(msg)
\textcolor{ansi-green-intense}{\textbf{    157}} \def\tcRGB{\textcolor[RGB]}\expandafter\tcRGB\expandafter{\detokenize{0,135,0}}{\textbf{return}} \def\tcRGB{\textcolor[RGB]}\expandafter\tcRGB\expandafter{\detokenize{0,135,0}}{\textbf{False}}  \def\tcRGB{\textcolor[RGB]}\expandafter\tcRGB\expandafter{\detokenize{95,135,135}}{\# failure}

\textcolor{ansi-red}{LatexFailed}: PDF creating failed, captured latex output:
Failed to run "['xelatex', 'notebook.tex', '-quiet']" command:
This is XeTeX, Version 3.141592653-2.6-0.999995 (TeX Live 2023) (preloaded format=xelatex)
 restricted \textbackslash{}write18 enabled.
entering extended mode
(./notebook.tex
LaTeX2e <2023-11-01> patch level 1
L3 programming layer <2024-01-22>
(/usr/local/texlive/2023basic/texmf-dist/tex/latex/base/article.cls
Document Class: article 2023/05/17 v1.4n Standard LaTeX document class
(/usr/local/texlive/2023basic/texmf-dist/tex/latex/base/size10.clo))
(/usr/local/texlive/2023basic/texmf-dist/tex/latex/graphics/graphicx.sty
(/usr/local/texlive/2023basic/texmf-dist/tex/latex/graphics/keyval.sty)
(/usr/local/texlive/2023basic/texmf-dist/tex/latex/graphics/graphics.sty
(/usr/local/texlive/2023basic/texmf-dist/tex/latex/graphics/trig.sty)
(/usr/local/texlive/2023basic/texmf-dist/tex/latex/graphics-cfg/graphics.cfg)
(/usr/local/texlive/2023basic/texmf-dist/tex/latex/graphics-def/xetex.def)))
(/usr/local/texlive/2023basic/texmf-dist/tex/latex/caption/caption.sty
(/usr/local/texlive/2023basic/texmf-dist/tex/latex/caption/caption3.sty))
(/usr/local/texlive/2023basic/texmf-dist/tex/latex/float/float.sty)
(/usr/local/texlive/2023basic/texmf-dist/tex/latex/xcolor/xcolor.sty
(/usr/local/texlive/2023basic/texmf-dist/tex/latex/graphics-cfg/color.cfg)
(/usr/local/texlive/2023basic/texmf-dist/tex/latex/graphics/mathcolor.ltx))
(/usr/local/texlive/2023basic/texmf-dist/tex/latex/tools/enumerate.sty)
(/usr/local/texlive/2023basic/texmf-dist/tex/latex/geometry/geometry.sty
(/usr/local/texlive/2023basic/texmf-dist/tex/generic/iftex/ifvtex.sty
(/usr/local/texlive/2023basic/texmf-dist/tex/generic/iftex/iftex.sty)))
(/usr/local/texlive/2023basic/texmf-dist/tex/latex/amsmath/amsmath.sty
For additional information on amsmath, use the `?' option.
(/usr/local/texlive/2023basic/texmf-dist/tex/latex/amsmath/amstext.sty
(/usr/local/texlive/2023basic/texmf-dist/tex/latex/amsmath/amsgen.sty))
(/usr/local/texlive/2023basic/texmf-dist/tex/latex/amsmath/amsbsy.sty)
(/usr/local/texlive/2023basic/texmf-dist/tex/latex/amsmath/amsopn.sty))
(/usr/local/texlive/2023basic/texmf-dist/tex/latex/amsfonts/amssymb.sty
(/usr/local/texlive/2023basic/texmf-dist/tex/latex/amsfonts/amsfonts.sty))
(/usr/local/texlive/2023basic/texmf-dist/tex/latex/base/textcomp.sty)
(/usr/local/texlive/2023basic/texmf-dist/tex/latex/upquote/upquote.sty)
(/usr/local/texlive/2023basic/texmf-dist/tex/latex/eurosym/eurosym.sty)
(/usr/local/texlive/2023basic/texmf-dist/tex/latex/fontspec/fontspec.sty
(/usr/local/texlive/2023basic/texmf-dist/tex/latex/l3packages/xparse/xparse.sty
(/usr/local/texlive/2023basic/texmf-dist/tex/latex/l3kernel/expl3.sty
(/usr/local/texlive/2023basic/texmf-dist/tex/latex/l3backend/l3backend-xetex.de
f)))
(/usr/local/texlive/2023basic/texmf-dist/tex/latex/fontspec/fontspec-xetex.sty
(/usr/local/texlive/2023basic/texmf-dist/tex/latex/base/fontenc.sty)
(/usr/local/texlive/2023basic/texmf-dist/tex/latex/fontspec/fontspec.cfg)))
(/usr/local/texlive/2023basic/texmf-dist/tex/latex/unicode-math/unicode-math.st
y
(/usr/local/texlive/2023basic/texmf-dist/tex/latex/unicode-math/unicode-math-xe
tex.sty
(/usr/local/texlive/2023basic/texmf-dist/tex/latex/l3packages/l3keys2e/l3keys2e
.sty) (/usr/local/texlive/2023basic/texmf-dist/tex/latex/base/fix-cm.sty
(/usr/local/texlive/2023basic/texmf-dist/tex/latex/base/ts1enc.def))
(/usr/local/texlive/2023basic/texmf-dist/tex/latex/unicode-math/unicode-math-ta
ble.tex)))
(/usr/local/texlive/2023basic/texmf-dist/tex/latex/fancyvrb/fancyvrb.sty)
(/usr/local/texlive/2023basic/texmf-dist/tex/latex/grffile/grffile.sty)

! LaTeX Error: File `adjustbox.sty' not found.

Type X to quit or <RETURN> to proceed,
or enter new name. (Default extension: sty)

Enter file name: 
! Emergency stop.
<read *> 
         
l.80     \textbackslash{}adjustboxset
                      \{max size=\{0.9\textbackslash{}linewidth\}\{0.9\textbackslash{}paperheight\}\}\^{}\^{}M
No pages of output.
Transcript written on notebook.log.


During handling of the above exception, another exception occurred:

\textcolor{ansi-red}{AttributeError}                            Traceback (most recent call last)
Cell \textcolor{ansi-green}{In[38], line 2}
\textcolor{ansi-green-intense}{\textbf{      1}} \def\tcRGB{\textcolor[RGB]}\expandafter\tcRGB\expandafter{\detokenize{95,135,135}}{\# Save your notebook first, then run this cell to export your submission.}
\textcolor{ansi-green}{----> 2} \setlength{\fboxsep}{0pt}\colorbox{ansi-yellow}{grader\strut}\def\tcRGB{\textcolor[RGB]}\expandafter\tcRGB\expandafter{\detokenize{98,98,98}}{\setlength{\fboxsep}{0pt}\colorbox{ansi-yellow}{.\strut}}\setlength{\fboxsep}{0pt}\colorbox{ansi-yellow}{export\strut}\setlength{\fboxsep}{0pt}\colorbox{ansi-yellow}{(\strut}\setlength{\fboxsep}{0pt}\colorbox{ansi-yellow}{run\_tests\strut}\def\tcRGB{\textcolor[RGB]}\expandafter\tcRGB\expandafter{\detokenize{98,98,98}}{\setlength{\fboxsep}{0pt}\colorbox{ansi-yellow}{=\strut}}\def\tcRGB{\textcolor[RGB]}\expandafter\tcRGB\expandafter{\detokenize{0,135,0}}{\setlength{\fboxsep}{0pt}\colorbox{ansi-yellow}{\textbf{True}\strut}}\setlength{\fboxsep}{0pt}\colorbox{ansi-yellow}{)\strut}

File \textcolor{ansi-green}{\textasciitilde{}/Library/Python/3.9/lib/python/site-packages/otter/check/utils.py:184}, in \textcolor{ansi-cyan}{grading\_mode\_disabled}\textcolor{ansi-blue}{(wrapped, self, args, kwargs)}
\textcolor{ansi-green-intense}{\textbf{    182}} \def\tcRGB{\textcolor[RGB]}\expandafter\tcRGB\expandafter{\detokenize{0,135,0}}{\textbf{if}} \def\tcRGB{\textcolor[RGB]}\expandafter\tcRGB\expandafter{\detokenize{0,135,0}}{type}(\def\tcRGB{\textcolor[RGB]}\expandafter\tcRGB\expandafter{\detokenize{0,135,0}}{self})\def\tcRGB{\textcolor[RGB]}\expandafter\tcRGB\expandafter{\detokenize{98,98,98}}{.}\_grading\_mode:
\textcolor{ansi-green-intense}{\textbf{    183}}     \def\tcRGB{\textcolor[RGB]}\expandafter\tcRGB\expandafter{\detokenize{0,135,0}}{\textbf{return}}
\textcolor{ansi-green}{--> 184} \def\tcRGB{\textcolor[RGB]}\expandafter\tcRGB\expandafter{\detokenize{0,135,0}}{\textbf{return}} \setlength{\fboxsep}{0pt}\colorbox{ansi-yellow}{wrapped\strut}\setlength{\fboxsep}{0pt}\colorbox{ansi-yellow}{(\strut}\def\tcRGB{\textcolor[RGB]}\expandafter\tcRGB\expandafter{\detokenize{98,98,98}}{\setlength{\fboxsep}{0pt}\colorbox{ansi-yellow}{*\strut}}\setlength{\fboxsep}{0pt}\colorbox{ansi-yellow}{args\strut}\setlength{\fboxsep}{0pt}\colorbox{ansi-yellow}{,\strut}\setlength{\fboxsep}{0pt}\colorbox{ansi-yellow}{ \strut}\def\tcRGB{\textcolor[RGB]}\expandafter\tcRGB\expandafter{\detokenize{98,98,98}}{\setlength{\fboxsep}{0pt}\colorbox{ansi-yellow}{*\strut}}\def\tcRGB{\textcolor[RGB]}\expandafter\tcRGB\expandafter{\detokenize{98,98,98}}{\setlength{\fboxsep}{0pt}\colorbox{ansi-yellow}{*\strut}}\setlength{\fboxsep}{0pt}\colorbox{ansi-yellow}{kwargs\strut}\setlength{\fboxsep}{0pt}\colorbox{ansi-yellow}{)\strut}

File \textcolor{ansi-green}{\textasciitilde{}/Library/Python/3.9/lib/python/site-packages/otter/check/utils.py:166}, in \textcolor{ansi-cyan}{incompatible\_with.<locals>.incompatible}\textcolor{ansi-blue}{(wrapped, self, args, kwargs)}
\textcolor{ansi-green-intense}{\textbf{    164}}     \def\tcRGB{\textcolor[RGB]}\expandafter\tcRGB\expandafter{\detokenize{0,135,0}}{\textbf{else}}:
\textcolor{ansi-green-intense}{\textbf{    165}}         \def\tcRGB{\textcolor[RGB]}\expandafter\tcRGB\expandafter{\detokenize{0,135,0}}{\textbf{return}}
\textcolor{ansi-green}{--> 166} \def\tcRGB{\textcolor[RGB]}\expandafter\tcRGB\expandafter{\detokenize{0,135,0}}{\textbf{return}} \setlength{\fboxsep}{0pt}\colorbox{ansi-yellow}{wrapped\strut}\setlength{\fboxsep}{0pt}\colorbox{ansi-yellow}{(\strut}\def\tcRGB{\textcolor[RGB]}\expandafter\tcRGB\expandafter{\detokenize{98,98,98}}{\setlength{\fboxsep}{0pt}\colorbox{ansi-yellow}{*\strut}}\setlength{\fboxsep}{0pt}\colorbox{ansi-yellow}{args\strut}\setlength{\fboxsep}{0pt}\colorbox{ansi-yellow}{,\strut}\setlength{\fboxsep}{0pt}\colorbox{ansi-yellow}{ \strut}\def\tcRGB{\textcolor[RGB]}\expandafter\tcRGB\expandafter{\detokenize{98,98,98}}{\setlength{\fboxsep}{0pt}\colorbox{ansi-yellow}{*\strut}}\def\tcRGB{\textcolor[RGB]}\expandafter\tcRGB\expandafter{\detokenize{98,98,98}}{\setlength{\fboxsep}{0pt}\colorbox{ansi-yellow}{*\strut}}\setlength{\fboxsep}{0pt}\colorbox{ansi-yellow}{kwargs\strut}\setlength{\fboxsep}{0pt}\colorbox{ansi-yellow}{)\strut}

File \textcolor{ansi-green}{\textasciitilde{}/Library/Python/3.9/lib/python/site-packages/otter/check/utils.py:217}, in \textcolor{ansi-cyan}{logs\_event.<locals>.event\_logger}\textcolor{ansi-blue}{(wrapped, self, args, kwargs)}
\textcolor{ansi-green-intense}{\textbf{    215}} \def\tcRGB{\textcolor[RGB]}\expandafter\tcRGB\expandafter{\detokenize{0,135,0}}{\textbf{except}} \def\tcRGB{\textcolor[RGB]}\expandafter\tcRGB\expandafter{\detokenize{215,95,95}}{\textbf{Exception}} \def\tcRGB{\textcolor[RGB]}\expandafter\tcRGB\expandafter{\detokenize{0,135,0}}{\textbf{as}} e:
\textcolor{ansi-green-intense}{\textbf{    216}}     \def\tcRGB{\textcolor[RGB]}\expandafter\tcRGB\expandafter{\detokenize{0,135,0}}{self}\def\tcRGB{\textcolor[RGB]}\expandafter\tcRGB\expandafter{\detokenize{98,98,98}}{.}\_log\_event(event\_type, success\def\tcRGB{\textcolor[RGB]}\expandafter\tcRGB\expandafter{\detokenize{98,98,98}}{=}\def\tcRGB{\textcolor[RGB]}\expandafter\tcRGB\expandafter{\detokenize{0,135,0}}{\textbf{False}}, error\def\tcRGB{\textcolor[RGB]}\expandafter\tcRGB\expandafter{\detokenize{98,98,98}}{=}e)
\textcolor{ansi-green}{--> 217}     \def\tcRGB{\textcolor[RGB]}\expandafter\tcRGB\expandafter{\detokenize{0,135,0}}{\textbf{raise}} e
\textcolor{ansi-green-intense}{\textbf{    219}} \def\tcRGB{\textcolor[RGB]}\expandafter\tcRGB\expandafter{\detokenize{0,135,0}}{\textbf{if}} ret \def\tcRGB{\textcolor[RGB]}\expandafter\tcRGB\expandafter{\detokenize{175,0,255}}{\textbf{is}} \def\tcRGB{\textcolor[RGB]}\expandafter\tcRGB\expandafter{\detokenize{0,135,0}}{\textbf{None}}:
\textcolor{ansi-green-intense}{\textbf{    220}}     ret \def\tcRGB{\textcolor[RGB]}\expandafter\tcRGB\expandafter{\detokenize{98,98,98}}{=} LoggedEventReturnValue(\def\tcRGB{\textcolor[RGB]}\expandafter\tcRGB\expandafter{\detokenize{0,135,0}}{\textbf{None}})

File \textcolor{ansi-green}{\textasciitilde{}/Library/Python/3.9/lib/python/site-packages/otter/check/utils.py:213}, in \textcolor{ansi-cyan}{logs\_event.<locals>.event\_logger}\textcolor{ansi-blue}{(wrapped, self, args, kwargs)}
\textcolor{ansi-green-intense}{\textbf{    208}} \def\tcRGB{\textcolor[RGB]}\expandafter\tcRGB\expandafter{\detokenize{175,0,0}}{"""}
\textcolor{ansi-green-intense}{\textbf{    209}} \def\tcRGB{\textcolor[RGB]}\expandafter\tcRGB\expandafter{\detokenize{175,0,0}}{Runs a method, catching any errors and logging the call. Returns the unwrapped return value}
\textcolor{ansi-green-intense}{\textbf{    210}} \def\tcRGB{\textcolor[RGB]}\expandafter\tcRGB\expandafter{\detokenize{175,0,0}}{of the wrapped function.}
\textcolor{ansi-green-intense}{\textbf{    211}} \def\tcRGB{\textcolor[RGB]}\expandafter\tcRGB\expandafter{\detokenize{175,0,0}}{"""}
\textcolor{ansi-green-intense}{\textbf{    212}} \def\tcRGB{\textcolor[RGB]}\expandafter\tcRGB\expandafter{\detokenize{0,135,0}}{\textbf{try}}:
\textcolor{ansi-green}{--> 213}     ret: Optional[LoggedEventReturnValue[T]] \def\tcRGB{\textcolor[RGB]}\expandafter\tcRGB\expandafter{\detokenize{98,98,98}}{=} \setlength{\fboxsep}{0pt}\colorbox{ansi-yellow}{wrapped\strut}\setlength{\fboxsep}{0pt}\colorbox{ansi-yellow}{(\strut}\def\tcRGB{\textcolor[RGB]}\expandafter\tcRGB\expandafter{\detokenize{98,98,98}}{\setlength{\fboxsep}{0pt}\colorbox{ansi-yellow}{*\strut}}\setlength{\fboxsep}{0pt}\colorbox{ansi-yellow}{args\strut}\setlength{\fboxsep}{0pt}\colorbox{ansi-yellow}{,\strut}\setlength{\fboxsep}{0pt}\colorbox{ansi-yellow}{ \strut}\def\tcRGB{\textcolor[RGB]}\expandafter\tcRGB\expandafter{\detokenize{98,98,98}}{\setlength{\fboxsep}{0pt}\colorbox{ansi-yellow}{*\strut}}\def\tcRGB{\textcolor[RGB]}\expandafter\tcRGB\expandafter{\detokenize{98,98,98}}{\setlength{\fboxsep}{0pt}\colorbox{ansi-yellow}{*\strut}}\setlength{\fboxsep}{0pt}\colorbox{ansi-yellow}{kwargs\strut}\setlength{\fboxsep}{0pt}\colorbox{ansi-yellow}{)\strut}
\textcolor{ansi-green-intense}{\textbf{    215}} \def\tcRGB{\textcolor[RGB]}\expandafter\tcRGB\expandafter{\detokenize{0,135,0}}{\textbf{except}} \def\tcRGB{\textcolor[RGB]}\expandafter\tcRGB\expandafter{\detokenize{215,95,95}}{\textbf{Exception}} \def\tcRGB{\textcolor[RGB]}\expandafter\tcRGB\expandafter{\detokenize{0,135,0}}{\textbf{as}} e:
\textcolor{ansi-green-intense}{\textbf{    216}}     \def\tcRGB{\textcolor[RGB]}\expandafter\tcRGB\expandafter{\detokenize{0,135,0}}{self}\def\tcRGB{\textcolor[RGB]}\expandafter\tcRGB\expandafter{\detokenize{98,98,98}}{.}\_log\_event(event\_type, success\def\tcRGB{\textcolor[RGB]}\expandafter\tcRGB\expandafter{\detokenize{98,98,98}}{=}\def\tcRGB{\textcolor[RGB]}\expandafter\tcRGB\expandafter{\detokenize{0,135,0}}{\textbf{False}}, error\def\tcRGB{\textcolor[RGB]}\expandafter\tcRGB\expandafter{\detokenize{98,98,98}}{=}e)

File \textcolor{ansi-green}{\textasciitilde{}/Library/Python/3.9/lib/python/site-packages/otter/check/notebook.py:525}, in \textcolor{ansi-cyan}{Notebook.export}\textcolor{ansi-blue}{(self, nb\_path, export\_path, pdf, filtering, pagebreaks, files, display\_link, force\_save, run\_tests)}
\textcolor{ansi-green-intense}{\textbf{    523}} \def\tcRGB{\textcolor[RGB]}\expandafter\tcRGB\expandafter{\detokenize{0,135,0}}{\textbf{if}} pdf\_created \def\tcRGB{\textcolor[RGB]}\expandafter\tcRGB\expandafter{\detokenize{175,0,255}}{\textbf{or}} \def\tcRGB{\textcolor[RGB]}\expandafter\tcRGB\expandafter{\detokenize{175,0,255}}{\textbf{not}} \def\tcRGB{\textcolor[RGB]}\expandafter\tcRGB\expandafter{\detokenize{0,135,0}}{self}\def\tcRGB{\textcolor[RGB]}\expandafter\tcRGB\expandafter{\detokenize{98,98,98}}{.}\_nbmeta\_config\def\tcRGB{\textcolor[RGB]}\expandafter\tcRGB\expandafter{\detokenize{98,98,98}}{.}require\_no\_pdf\_confirmation:
\textcolor{ansi-green-intense}{\textbf{    524}}     \def\tcRGB{\textcolor[RGB]}\expandafter\tcRGB\expandafter{\detokenize{0,135,0}}{\textbf{if}} pdf\_error \def\tcRGB{\textcolor[RGB]}\expandafter\tcRGB\expandafter{\detokenize{175,0,255}}{\textbf{is}} \def\tcRGB{\textcolor[RGB]}\expandafter\tcRGB\expandafter{\detokenize{175,0,255}}{\textbf{not}} \def\tcRGB{\textcolor[RGB]}\expandafter\tcRGB\expandafter{\detokenize{0,135,0}}{\textbf{None}}:
\textcolor{ansi-green}{--> 525}         \def\tcRGB{\textcolor[RGB]}\expandafter\tcRGB\expandafter{\detokenize{0,135,0}}{\textbf{raise}} pdf\_error
\textcolor{ansi-green-intense}{\textbf{    526}}     continue\_export()
\textcolor{ansi-green-intense}{\textbf{    527}} \def\tcRGB{\textcolor[RGB]}\expandafter\tcRGB\expandafter{\detokenize{0,135,0}}{\textbf{else}}:

File \textcolor{ansi-green}{\textasciitilde{}/Library/Python/3.9/lib/python/site-packages/otter/check/notebook.py:462}, in \textcolor{ansi-cyan}{Notebook.export}\textcolor{ansi-blue}{(self, nb\_path, export\_path, pdf, filtering, pagebreaks, files, display\_link, force\_save, run\_tests)}
\textcolor{ansi-green-intense}{\textbf{    460}} pdf\_path, pdf\_created, pdf\_error \def\tcRGB{\textcolor[RGB]}\expandafter\tcRGB\expandafter{\detokenize{98,98,98}}{=} \def\tcRGB{\textcolor[RGB]}\expandafter\tcRGB\expandafter{\detokenize{0,135,0}}{\textbf{None}}, \def\tcRGB{\textcolor[RGB]}\expandafter\tcRGB\expandafter{\detokenize{0,135,0}}{\textbf{True}}, \def\tcRGB{\textcolor[RGB]}\expandafter\tcRGB\expandafter{\detokenize{0,135,0}}{\textbf{None}}
\textcolor{ansi-green-intense}{\textbf{    461}} \def\tcRGB{\textcolor[RGB]}\expandafter\tcRGB\expandafter{\detokenize{0,135,0}}{\textbf{if}} pdf:
\textcolor{ansi-green}{--> 462}     \def\tcRGB{\textcolor[RGB]}\expandafter\tcRGB\expandafter{\detokenize{0,135,0}}{\textbf{try}}: pdf\_path \def\tcRGB{\textcolor[RGB]}\expandafter\tcRGB\expandafter{\detokenize{98,98,98}}{=} \setlength{\fboxsep}{0pt}\colorbox{ansi-yellow}{export\_notebook\strut}\setlength{\fboxsep}{0pt}\colorbox{ansi-yellow}{(\strut}\setlength{\fboxsep}{0pt}\colorbox{ansi-yellow}{nb\_path\strut}\setlength{\fboxsep}{0pt}\colorbox{ansi-yellow}{,\strut}\setlength{\fboxsep}{0pt}\colorbox{ansi-yellow}{ \strut}\setlength{\fboxsep}{0pt}\colorbox{ansi-yellow}{filtering\strut}\def\tcRGB{\textcolor[RGB]}\expandafter\tcRGB\expandafter{\detokenize{98,98,98}}{\setlength{\fboxsep}{0pt}\colorbox{ansi-yellow}{=\strut}}\setlength{\fboxsep}{0pt}\colorbox{ansi-yellow}{filtering\strut}\setlength{\fboxsep}{0pt}\colorbox{ansi-yellow}{,\strut}\setlength{\fboxsep}{0pt}\colorbox{ansi-yellow}{ \strut}\setlength{\fboxsep}{0pt}\colorbox{ansi-yellow}{pagebreaks\strut}\def\tcRGB{\textcolor[RGB]}\expandafter\tcRGB\expandafter{\detokenize{98,98,98}}{\setlength{\fboxsep}{0pt}\colorbox{ansi-yellow}{=\strut}}\setlength{\fboxsep}{0pt}\colorbox{ansi-yellow}{pagebreaks\strut}\setlength{\fboxsep}{0pt}\colorbox{ansi-yellow}{)\strut}
\textcolor{ansi-green-intense}{\textbf{    463}}     \def\tcRGB{\textcolor[RGB]}\expandafter\tcRGB\expandafter{\detokenize{0,135,0}}{\textbf{except}} \def\tcRGB{\textcolor[RGB]}\expandafter\tcRGB\expandafter{\detokenize{215,95,95}}{\textbf{Exception}} \def\tcRGB{\textcolor[RGB]}\expandafter\tcRGB\expandafter{\detokenize{0,135,0}}{\textbf{as}} e: pdf\_error \def\tcRGB{\textcolor[RGB]}\expandafter\tcRGB\expandafter{\detokenize{98,98,98}}{=} e
\textcolor{ansi-green-intense}{\textbf{    464}}     \def\tcRGB{\textcolor[RGB]}\expandafter\tcRGB\expandafter{\detokenize{0,135,0}}{\textbf{if}} pdf\_path \def\tcRGB{\textcolor[RGB]}\expandafter\tcRGB\expandafter{\detokenize{175,0,255}}{\textbf{and}} os\def\tcRGB{\textcolor[RGB]}\expandafter\tcRGB\expandafter{\detokenize{98,98,98}}{.}path\def\tcRGB{\textcolor[RGB]}\expandafter\tcRGB\expandafter{\detokenize{98,98,98}}{.}isfile(pdf\_path):

File \textcolor{ansi-green}{\textasciitilde{}/Library/Python/3.9/lib/python/site-packages/otter/export/\_\_init\_\_.py:36}, in \textcolor{ansi-cyan}{export\_notebook}\textcolor{ansi-blue}{(nb\_path, dest, exporter\_type, **kwargs)}
\textcolor{ansi-green-intense}{\textbf{     33}}     pdf\_name \def\tcRGB{\textcolor[RGB]}\expandafter\tcRGB\expandafter{\detokenize{98,98,98}}{=} os\def\tcRGB{\textcolor[RGB]}\expandafter\tcRGB\expandafter{\detokenize{98,98,98}}{.}path\def\tcRGB{\textcolor[RGB]}\expandafter\tcRGB\expandafter{\detokenize{98,98,98}}{.}splitext(nb\_path)[\def\tcRGB{\textcolor[RGB]}\expandafter\tcRGB\expandafter{\detokenize{98,98,98}}{0}] \def\tcRGB{\textcolor[RGB]}\expandafter\tcRGB\expandafter{\detokenize{98,98,98}}{+} \def\tcRGB{\textcolor[RGB]}\expandafter\tcRGB\expandafter{\detokenize{175,0,0}}{"}\def\tcRGB{\textcolor[RGB]}\expandafter\tcRGB\expandafter{\detokenize{175,0,0}}{.pdf}\def\tcRGB{\textcolor[RGB]}\expandafter\tcRGB\expandafter{\detokenize{175,0,0}}{"}
\textcolor{ansi-green-intense}{\textbf{     35}} Exporter \def\tcRGB{\textcolor[RGB]}\expandafter\tcRGB\expandafter{\detokenize{98,98,98}}{=} get\_exporter(exporter\_type\def\tcRGB{\textcolor[RGB]}\expandafter\tcRGB\expandafter{\detokenize{98,98,98}}{=}exporter\_type)
\textcolor{ansi-green}{---> 36} \setlength{\fboxsep}{0pt}\colorbox{ansi-yellow}{Exporter\strut}\def\tcRGB{\textcolor[RGB]}\expandafter\tcRGB\expandafter{\detokenize{98,98,98}}{\setlength{\fboxsep}{0pt}\colorbox{ansi-yellow}{.\strut}}\setlength{\fboxsep}{0pt}\colorbox{ansi-yellow}{convert\_notebook\strut}\setlength{\fboxsep}{0pt}\colorbox{ansi-yellow}{(\strut}\setlength{\fboxsep}{0pt}\colorbox{ansi-yellow}{nb\_path\strut}\setlength{\fboxsep}{0pt}\colorbox{ansi-yellow}{,\strut}\setlength{\fboxsep}{0pt}\colorbox{ansi-yellow}{ \strut}\setlength{\fboxsep}{0pt}\colorbox{ansi-yellow}{pdf\_name\strut}\setlength{\fboxsep}{0pt}\colorbox{ansi-yellow}{,\strut}\setlength{\fboxsep}{0pt}\colorbox{ansi-yellow}{ \strut}\def\tcRGB{\textcolor[RGB]}\expandafter\tcRGB\expandafter{\detokenize{98,98,98}}{\setlength{\fboxsep}{0pt}\colorbox{ansi-yellow}{*\strut}}\def\tcRGB{\textcolor[RGB]}\expandafter\tcRGB\expandafter{\detokenize{98,98,98}}{\setlength{\fboxsep}{0pt}\colorbox{ansi-yellow}{*\strut}}\setlength{\fboxsep}{0pt}\colorbox{ansi-yellow}{kwargs\strut}\setlength{\fboxsep}{0pt}\colorbox{ansi-yellow}{)\strut}
\textcolor{ansi-green-intense}{\textbf{     38}} \def\tcRGB{\textcolor[RGB]}\expandafter\tcRGB\expandafter{\detokenize{0,135,0}}{\textbf{return}} pdf\_name

File \textcolor{ansi-green}{\textasciitilde{}/Library/Python/3.9/lib/python/site-packages/otter/export/exporters/via\_latex.py:70}, in \textcolor{ansi-cyan}{PDFViaLatexExporter.convert\_notebook}\textcolor{ansi-blue}{(cls, nb\_path, dest, xecjk, **kwargs)}
\textcolor{ansi-green-intense}{\textbf{     67}}     \def\tcRGB{\textcolor[RGB]}\expandafter\tcRGB\expandafter{\detokenize{0,135,0}}{\textbf{with}} \def\tcRGB{\textcolor[RGB]}\expandafter\tcRGB\expandafter{\detokenize{0,135,0}}{open}(dest, \def\tcRGB{\textcolor[RGB]}\expandafter\tcRGB\expandafter{\detokenize{175,0,0}}{"}\def\tcRGB{\textcolor[RGB]}\expandafter\tcRGB\expandafter{\detokenize{175,0,0}}{wb}\def\tcRGB{\textcolor[RGB]}\expandafter\tcRGB\expandafter{\detokenize{175,0,0}}{"}) \def\tcRGB{\textcolor[RGB]}\expandafter\tcRGB\expandafter{\detokenize{0,135,0}}{\textbf{as}} output\_file:
\textcolor{ansi-green-intense}{\textbf{     68}}         output\_file\def\tcRGB{\textcolor[RGB]}\expandafter\tcRGB\expandafter{\detokenize{98,98,98}}{.}write(pdf\_output[\def\tcRGB{\textcolor[RGB]}\expandafter\tcRGB\expandafter{\detokenize{98,98,98}}{0}])
\textcolor{ansi-green}{---> 70} \def\tcRGB{\textcolor[RGB]}\expandafter\tcRGB\expandafter{\detokenize{0,135,0}}{\textbf{except}} \setlength{\fboxsep}{0pt}\colorbox{ansi-yellow}{nbconvert\strut}\def\tcRGB{\textcolor[RGB]}\expandafter\tcRGB\expandafter{\detokenize{98,98,98}}{\setlength{\fboxsep}{0pt}\colorbox{ansi-yellow}{.\strut}}\setlength{\fboxsep}{0pt}\colorbox{ansi-yellow}{pdf\strut}\def\tcRGB{\textcolor[RGB]}\expandafter\tcRGB\expandafter{\detokenize{98,98,98}}{.}LatexFailed \def\tcRGB{\textcolor[RGB]}\expandafter\tcRGB\expandafter{\detokenize{0,135,0}}{\textbf{as}} error:
\textcolor{ansi-green-intense}{\textbf{     71}}     message \def\tcRGB{\textcolor[RGB]}\expandafter\tcRGB\expandafter{\detokenize{98,98,98}}{=} \def\tcRGB{\textcolor[RGB]}\expandafter\tcRGB\expandafter{\detokenize{175,0,0}}{"}\def\tcRGB{\textcolor[RGB]}\expandafter\tcRGB\expandafter{\detokenize{175,0,0}}{There was an error generating your LaTeX; showing full error message:}\def\tcRGB{\textcolor[RGB]}\expandafter\tcRGB\expandafter{\detokenize{175,95,0}}{\textbf{\textbackslash{}n}}\def\tcRGB{\textcolor[RGB]}\expandafter\tcRGB\expandafter{\detokenize{175,0,0}}{"}
\textcolor{ansi-green-intense}{\textbf{     72}}     message \def\tcRGB{\textcolor[RGB]}\expandafter\tcRGB\expandafter{\detokenize{98,98,98}}{+}\def\tcRGB{\textcolor[RGB]}\expandafter\tcRGB\expandafter{\detokenize{98,98,98}}{=} indent(error\def\tcRGB{\textcolor[RGB]}\expandafter\tcRGB\expandafter{\detokenize{98,98,98}}{.}output, \def\tcRGB{\textcolor[RGB]}\expandafter\tcRGB\expandafter{\detokenize{175,0,0}}{"}\def\tcRGB{\textcolor[RGB]}\expandafter\tcRGB\expandafter{\detokenize{175,0,0}}{    }\def\tcRGB{\textcolor[RGB]}\expandafter\tcRGB\expandafter{\detokenize{175,0,0}}{"})

\textcolor{ansi-red}{AttributeError}: module 'nbconvert' has no attribute 'pdf'
    \end{Verbatim}

    


    % Add a bibliography block to the postdoc
    
    
    
\end{document}
